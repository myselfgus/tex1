\chapter{O Jardim}

Voltando mais uma vez ao início, nosso Relojoeiro reabre as cortinas da oficina com um olhar que agora mistura a curiosidade de um explorador e o cansaço de um velho marinheiro, de um navegante que atravessou oceanos de possibilidades sem encontrar terra firme, ou talvez tenha encontrado terras demais, ilhas de verdades parciais que não formam um continente, um arquipélago de revelações fragmentadas que recusam organizar-se em um todo coerente, e mesmo assim ele persiste, pois que outra escolha teria, sabendo o que agora sabe, vendo o que agora vê, sentindo o que agora sente, este homem de meia-idade com as mãos calejadas pelo manuseio das horas e o coração arranhado pelas garras invisíveis do tempo, que é tigre e cordeiro, algoz e vítima, veneno e remédio, ele sente que cada volta ao começo o aproxima mais das respostas, ou talvez o empurre mais fundo nas perguntas, mas quem pode distinguir entre aproximação e afastamento quando o próprio sentido de direção foi subvertido, quando o mapa e o território trocaram de lugar, quando quem procura é também o procurado, quando quem pergunta é também a resposta que aguarda ser formulada no silêncio entre as palavras?

Fechando a oficina novamente, ele segue em frente, desta vez em direção ao Jardim do Tempo Perdido, um lugar sobre o qual ouviu murmúrios, fragmentos de conversas, meias frases sussurradas como segredos culposos, um lugar que parece existir nas dobras da realidade, nos interstícios entre o que foi e o que poderia ter sido, nas fronteiras porosas entre a memória e o esquecimento, o conhecido e o irreconhecível, e enquanto caminha, o Relojoeiro sente uma estranha duplicação dentro de si, como se fosse não apenas um, mas muitos, não por fragmentação da personalidade como sugeriria algum diagnóstico apressado, mas por expansão, por multiplicação, como se cada passo o desdobrasse em versões alternativas que pensam pensamentos ligeiramente diferentes, sentem sentimentos sutilmente distintos, perguntam-se perguntas quase idênticas mas com entonações variadas, não sou apenas um, pensa ele, sou uma multidão, um parlamento de eus que debate sem chegar a consenso, uma democracia interna onde cada voz tem direito a voto mas nenhuma tem poder de veto, e não seria essa a melhor definição do que chamamos consciência, esse diálogo perpétuo entre as muitas máscaras que usamos não para esconder, mas para revelar diferentes aspectos do que somos ou poderíamos ser?

Ao se aproximar do jardim, o Relojoeiro nota uma mudança no ar, uma densidade que carrega o peso dos momentos passados, uma substância quase palpável que parece resistir a seus movimentos como água resistiria a um nadador, mas não é água, não é ar, não é matéria no sentido convencional, é algo mais sutil e ao mesmo tempo mais denso, mais essencial, como se o próprio espaço fosse saturado de temporalidade condensada, de passados cristalizados, de possíveis jamais realizados que, no entanto, mantêm uma existência espectral, fantasmática, nas margens do que chamamos real, árvores altas e retorcidas se erguem como guardiãs de segredos antigos, seus troncos marcados não por anéis de crescimento, mas por cicatrizes de eventos que nunca ocorreram, por incisões feitas por facas que nunca foram forjadas, por golpes desferidos por mãos que nunca nasceram, e ainda assim tão reais quanto qualquer coisa neste mundo de aparências enganosas, de certezas ilusórias, de verdades provisórias.

Um aroma de nostalgia permeia o ambiente, não um perfume ou um cheiro que poderia ser analisado por químicos, decomposto em moléculas, reduzido a fórmulas, mas uma presença que se dirige diretamente à alma, ao núcleo mais íntimo e indecifrável do ser, como se cada inspiração trouxesse não apenas oxigênio para os pulmões, mas também fragmentos de vidas não vividas para a consciência, estilhaços de sonhos abandonados para o coração, resíduos de esperanças frustradas para o espírito, o Relojoeiro atravessa o portão enferrujado, que range como se protestasse contra qualquer intruso, contra qualquer testemunha, como se o jardim preferisse permanecer não observado, não visitado, fechado em sua própria completude auto-suficiente, em seu próprio ciclo perpétuo de preservação e transformação, de lembrança e esquecimento, de morte e renascimento.

``Bem-vindo ao Jardim do Tempo Perdido,'' diz um velho jardineiro, materializado tão repentinamente que parece ter brotado do solo como uma de suas plantas, um homem cujas mãos calejadas conhecem o toque delicado do tempo, cujos dedos parecem capazes de sentir não apenas a textura das folhas e o pulso da seiva, mas também a consistência mais sutil das horas abandonadas, dos instantes descartados, dos momentos que escorregaram para fora da consciência coletiva como água entre os dedos, ``aqui cultivamos aquilo que foi esquecido, os momentos que passaram sem deixar rastro, mas que, de alguma forma, continuam a existir'', e o Relojoeiro, este homem que dedicou sua vida a medir o mensurável, a contar o contável, a dividir o divisível, sente uma vertigem diante dessa inversão fundamental, dessa agricultura do impalpável, dessa jardinagem do invisível, pois como se pode cultivar o esquecido quando o próprio ato de cultivo já implica em lembrar?

``E o que faz com essas memórias?'', pergunta o Relojoeiro, mais por educação do que por curiosidade, embora uma parte dele, aquela parte que nunca se satisfaz com explicações superficiais, com respostas convencionais, com verdades de segunda mão, esteja genuinamente intrigada, genuinamente estimulada por este paradoxo ambulante, este homem que colhe o que outros descartam, que preserva o que outros rejeitam, que valoriza precisamente aquilo que a sociedade, em sua obsessão pelo novo, pelo atual, pelo imediato, considera desprezível, dispensável, descartável, e o Relojoeiro pensa, não sem certa ironia, que este jardineiro é, à sua maneira, um revolucionário mais radical do que todos os insurgentes políticos, todos os visionários sociais, todos os reformadores culturais, pois desafia não apenas um sistema específico, mas a própria lógica do tempo linear, a própria tirania do presente sobre o passado.

``Cuidamos delas,'' responde o jardineiro, com a simplicidade de quem explica o óbvio, o auto-evidente, o inquestionável, ``para que nunca desapareçam completamente. Cada flor, cada árvore, representa um instante que alguém esqueceu, mas que o tempo, caprichoso como é, decidiu preservar'', e enquanto fala, suas mãos acariciam uma flor azul de formato impossível, uma estrutura de pétalas que parece desafiar as leis da geometria, da biologia, da física, uma criação que não poderia existir no mundo exterior, no mundo governado pela razão, pela causalidade, pela entropia, mas que aqui, neste jardim onde as regras são outras, onde a lógica é outra, onde a verdade é outra, floresce com uma vitalidade quase obscena, quase ultrajante, como um grito de cor em um universo monocromático, como uma afirmação de possibilidade em um reino de necessidades.

Caminhando pelo jardim, o Relojoeiro encontra árvores cujas folhas sussurram histórias esquecidas, histórias de amores não declarados, de palavras não ditas, de gestos não realizados, de caminhos não tomados, de vidas não vividas, cada folha uma narrativa completa em miniatura, um mundo possível compactado em clorofila e celulose, em luz solidificada e tempo cristalizado, ele para diante de uma pequena árvore que, curiosamente, parece familiar, como se a tivesse visto antes em um sonho, em uma memória, em uma premonição, ou talvez seja apenas o reconhecimento daquilo que sempre esteve dentro dele sem que soubesse, daquilo que sempre foi parte de sua estrutura mais profunda, de seu código mais íntimo, de sua verdade mais essencial.

``Esta é a sua árvore,'' diz o jardineiro, como quem revela um segredo que não deveria ser revelado, uma verdade que não deveria ser expressa, um conhecimento que deveria permanecer implícito, tácito, não articulado, ``cada folha é uma memória que você perdeu, cada ramo é uma parte de sua vida que você esqueceu'', e o Relojoeiro sente um arrepio percorrer sua espinha, uma corrente elétrica que conecta o cérebro às extremidades, o pensamento à sensação, o abstrato ao concreto, pois como pode existir uma representação física, material, tangível daquilo que, por definição, é ausente, é lacuna, é vazio? como pode o esquecido ser preservado sem deixar de ser esquecido? como pode a ausência ser tornada presença sem deixar de ser ausência?

O Relojoeiro toca uma folha, um gesto simples, banal, cotidiano, o tipo de contato com a natureza que normalmente não provocaria mais do que um registro sensorial efêmero, uma impressão tátil momentânea, uma experiência trivial rapidamente substituída por outra igualmente trivial na procissão interminável de estímulos que constitui a consciência ordinária, e no entanto, ao tocar esta folha específica, neste jardim específico, neste momento específico, ele é imediatamente transportado para uma tarde de verão de sua infância, um dia de risos e descobertas, um tempo que ele pensava estar perdido para sempre, mas que agora floresce diante de seus olhos com uma nitidez dolorosa, com uma clareza cortante, com uma imediatez que desafia as fronteiras entre passado e presente, entre memória e experiência, entre o que foi e o que é.

Ele vê-se novamente como criança, não como uma lembrança convencional, não como uma reconstrução mental baseada em fragmentos preservados, em narrativas repetidas, em fotografias desbotadas, mas como uma realidade alternativa, como uma bifurcação da linha temporal onde aquele momento nunca deixou de acontecer, onde continua ocorrendo em um eterno presente, em um agora expandido que contém todos os instantes simultaneamente, ele está de pé em um campo de centeio dourado como o sol, ao lado de seu avô, este homem que o introduziu aos mistérios dos mecanismos, às maravilhas das engrenagens, às magias da medição, o homem cujas mãos, tão semelhantes às suas próprias agora, lhe ensinaram que consertar relógios não era apenas um ofício, mas uma forma de comunhão com o tempo, uma maneira de participar do fluxo universal que conecta todas as coisas, todos os seres, todos os momentos em uma teia infinita de relações recíprocas.

``Veja,'' diz o avô, apontando para o horizonte onde o céu e a terra se encontram em uma linha tão definida que parece ter sido traçada com régua e tinta, ``o tempo não é o que pensamos que é. Não é uma flecha, não é uma linha, não é uma estrada. O tempo é um oceano, e nós somos apenas gotas nesse oceano, temporariamente separadas, destinadas a retornar'', e o menino que o Relojoeiro foi escuta com aquela atenção absoluta, aquela receptividade total que só as crianças possuem, antes que a educação as ensine a filtrar, a categorizar, a hierarquizar, a duvidar, a menino acredita porque ainda não aprendeu a desacreditar, aceita porque ainda não foi condicionado a rejeitar, compreende porque ainda não adquiriu a sofisticada ignorância que chamamos de conhecimento.

A experiência é ao mesmo tempo reconfortante e desconcertante, como encontrar um tesouro há muito perdido apenas para descobrir que ele nunca esteve realmente perdido, apenas temporariamente inacessível, temporariamente obscurecido por camadas de experiências posteriores, de conhecimentos adquiridos, de certezas construídas, o Relojoeiro percebe que a memória não é apenas uma função do cérebro, um processo neurológico, um fenômeno bioquímico, mas uma entidade viva, uma planta que precisa ser cuidada para não murchar e desaparecer, um organismo que respira, que cresce, que evolui, que se adapta, que resiste à entropia não por negá-la, mas por incorporá-la em sua própria natureza, em sua própria estrutura, em sua própria essência.

``Mas por que manter essas memórias vivas?'', pergunta ele, ainda atordoado pela experiência, ainda tentando reorientar-se em um universo que acaba de revelar-se muito mais complexo, muito mais misterioso, muito mais rico do que jamais suspeitara, um universo onde o descartado não desaparece, onde o perdido não se perde, onde o esquecido não se esvai, mas encontra um refúgio, um santuário, um jardim onde pode continuar existindo sob outra forma, sob outra manifestação, sob outra modalidade, ``não seria mais fácil deixá-las morrer?'', e esta pergunta, formulada com a ingenuidade de quem ainda pensa em termos de facilidade, de conveniência, de praticidade, revela o quanto ele ainda está preso ao paradigma da eficiência, ao modelo da utilidade, à lógica do custo-benefício que governa o mundo exterior, o mundo da produção e do consumo, o mundo da acumulação e do descarte.

O jardineiro sorri, um sorriso que carrega a sabedoria de séculos, um sorriso que transcende o individual, o pessoal, o biográfico, para expressar algo universal, algo arquetípico, algo que pertence não a um homem específico, mas à própria humanidade em sua jornada através do tempo, em sua luta contra o esquecimento, em sua resistência à dissolução, ``porque as memórias são a alma do tempo. Sem elas, o tempo é apenas uma sequência de eventos vazios, um desfile de horas sem significado. Preservamos as memórias para lembrar que cada momento, por mais insignificante que pareça, é parte de quem somos'', e nestas palavras simples, nesta declaração direta, nesta afirmação desprovida de ornamentos retóricos, de complexidades sintáticas, de sofisticações conceituais, o Relojoeiro encontra uma verdade tão fundamental, tão essencial, tão inescapável que se pergunta como pôde viver tantos anos sem reconhecê-la, sem honrá-la, sem incorporá-la à sua compreensão de si mesmo e do mundo.

Enquanto o Relojoeiro absorve essas palavras, ele vê uma mulher que parece estar perdida em suas próprias lembranças, uma figura de contornos suaves e expressão distante, como alguém que habita simultaneamente múltiplos planos de existência, múltiplas camadas de realidade, múltiplas dimensões de ser, ela olha para ele com uma mistura de tristeza e reconhecimento, como se estivesse vendo não apenas o homem que ele é agora, mas também o menino que ele foi, o adolescente que ele foi, o jovem que ele foi, e talvez também o velho que ele será, todas essas versões coexistindo em um único ponto focal, em um único momento de percepção expandida que transcende a linearidade, a sequencialidade, a unidirecionalidade do tempo convencional.

``Eu conheço você,'' diz ela, e sua voz tem a qualidade etérea de uma música ouvida em sonho, de uma conversa recordada através de paredes, de um eco que chega antes do som que o originou, ``nos encontramos aqui muitas vezes, mas você sempre esquece'', e nesta aparente contradição, neste paradoxo verbal, neste koan espontâneo, o Relojoeiro vislumbra uma possibilidade tão radical, tão revolucionária, tão transformadora que sua mente inicialmente recua, resiste, rejeita, porque aceitá-la significaria reconfigurar completamente sua compreensão não apenas do tempo, não apenas da memória, não apenas da consciência, mas da própria natureza da realidade, da própria estrutura do ser, da própria arquitetura do possível.

``Como posso esquecer algo tão importante?'', pergunta o Relojoeiro, intrigado por esta inversão da lógica habitual, por esta subversão da causalidade normal, por esta reorientação da sequência costumeira onde a importância de um evento é proporcional à sua memorabilidade, onde o significativo é, por definição, o que não pode ser esquecido, o que resiste ao desgaste temporal, o que permanece na consciência apesar da erosão contínua provocada pela passagem das horas, dos dias, dos anos, das décadas, e há em sua pergunta não apenas curiosidade intelectual, não apenas interesse conceitual, não apenas desejo de compreensão, mas também um elemento de angústia existencial, de inquietação ontológica, de desconforto metafísico diante da possibilidade de que aquilo que consideramos mais essencial, mais definidor, mais constitutivo de nossa identidade seja precisamente o que está mais sujeito ao esquecimento, à perda, à dissolução.

``Porque este é o Jardim do Tempo Perdido,'' responde ela, como se isso explicasse tudo, como se nessas poucas palavras estivesse contida toda a cosmologia, toda a epistemologia, toda a metafísica necessária para compreender as leis que governam este lugar, este espaço-tempo anômalo, este enclave de diferentes regras, diferentes princípios, diferentes realidades, ``e aqui, tudo que é importante tende a ser esquecido, para que possamos redescobrir constantemente'', e há nesta formulação uma poesia tão profunda, uma verdade tão condensada, uma sabedoria tão cristalina que o Relojoeiro sente como se algo dentro dele estivesse sendo simultaneamente destruído e reconstruído, desfeito e refeito, aniquilado e regenerado, como se suas estruturas internas, seus mapas conceituais, seus esquemas perceptivos estivessem sendo não apenas revisados ou atualizados, mas fundamentalmente transformados, metamorfoseados em algo completamente novo, completamente diferente, completamente outro.

Com essas palavras ecoando em sua mente, o Relojoeiro decide que é hora de partir, não porque tenha compreendido tudo, não porque tenha assimilado completamente a lição deste lugar estranho, este locus de contradições e paradoxos, este território onde as regras habituais da lógica, da causalidade, da temporalidade são simultaneamente observadas e subvertidas, mas porque intui que a próxima etapa de sua jornada o aguarda, que o próximo fragmento do quebra-cabeça existencial que tenta montar se encontra em outro lugar, que o próximo nível de compreensão exige uma nova paisagem, um novo contexto, um novo conjunto de desafios e revelações, agradece ao jardineiro e à mulher, prometendo a si mesmo que nunca mais esquecerá o que aprendeu ali, mesmo sabendo, com uma parte mais sábia de sua consciência, que tais promessas são tão inevitáveis quanto seu eventual descumprimento, tão necessárias quanto sua inescapável violação.

Com passos lentos, ele deixa o jardim, carregando consigo uma folha da sua árvore, um lembrete constante de que o tempo é feito de memórias, e que preservar essas memórias é preservar a própria essência da vida, esta mínima porção de matéria vegetal, esta estrutura celular aparentemente insignificante, este fragmento de clorofila e celulose torna-se, em suas mãos, em sua percepção, em sua consciência, algo infinitamente mais valioso do que qualquer jóia, qualquer metal precioso, qualquer artefato raro, pois contém não apenas um momento específico, uma experiência particular, uma vivência singular, mas o princípio mesmo da temporalidade significativa, da duração qualitativa, da continuidade essencial que transcende a mera sucessão de instantes, a simples acumulação de agoras, a pura adição de segundos, minutos, horas.

\bigskip

E assim, caro leitor, voltamos ao início, ao ponto onde o Relojoeiro, agora com uma folha de memória nas mãos e um coração mais leve, reabre as cortinas de sua oficina, e este movimento, este gesto aparentemente banal de revelar o interior de um espaço anteriormente oculto, ganha agora uma dimensão simbólica que antes lhe escapava, uma profundidade de significado que antes não percebia, uma ressonância cósmica que antes não sentia, porque agora ele sabe, com uma certeza que nenhuma dúvida pode abalar, que cada vez que abre uma cortina está também abrindo uma memória, cada vez que deixa a luz entrar está também deixando o passado retornar, cada vez que ilumina um espaço está também ressuscitando um tempo que parecia morto mas que apenas dormia, esperando ser acordado, esperando ser lembrado, esperando ser vivido novamente.

\begin{center}
\textit{A folha do Jardim do Tempo Perdido}
\end{center}
