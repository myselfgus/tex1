\chapter{O Horologista}

Voltando ao início uma última vez, o Relojoeiro reabre as cortinas da oficina, mas as mãos que realizam esse gesto já não lhe parecem próprias, como se pertencessem a outro homem, a outro tempo, a outra existência. A luz que invade o espaço não ilumina mais do que sombras sobre sombras, realidades palimpsestas sobrepostas em camadas de significado que já não consegue separar. Com um olhar que condensa tudo o que viu, tudo o que compreendeu e, principalmente, tudo o que jamais poderá compreender, ele contempla os cinco objetos dispostos sobre sua mesa de trabalho: a flor murcha da Cidade das Horas Invertidas, o fragmento de rocha do Vale da Imutabilidade, a folha do Jardim do Tempo Perdido, o cristal do Místico e a engrenagem quebrada da Revolução dos Ponteiros.

``Não são suficientes'', murmura para si mesmo, palavras que não arranham sequer a superfície do vazio que se expande dentro dele. ``Cinco fragmentos não bastam para reconstruir o universo. Cinco peças não são suficientes para consertar o tempo.''

Toma então uma decisão que já estava escrita no primeiro momento em que abriu as cortinas da oficina, no início desta jornada que só agora compreende não ter sido uma escolha, mas um destino inescapável, uma necessidade inscrita na própria estrutura do ser. Ele recolhe os cinco objetos, guarda-os em um pequeno saco de couro que prende ao cinto, e, pela última vez, fecha a porta da oficina. Desta vez, porém, não guarda a chave no bolso. Deixa-a na fechadura, girando lentamente até ouvir o clique final que separa definitivamente o que foi do que será.

Seu caminho agora é direto, sem hesitações, sem desvios. Não é mais um peregrino tateando no escuro, mas um homem que caminha com a certeza brutal dos que aceitaram seu fado. Seus pés o levam de volta ao Jardim do Tempo Perdido, atravessando paisagens que parecem simultaneamente familiares e estranhas, como se o mundo inteiro tivesse se transformado em um eco, uma reverberação de algo original que já não existe, que talvez nunca tenha existido.

O portão enferrujado range, mais alto agora, um protesto metálico contra sua intrusão, ou talvez um lamento pelo que está prestes a acontecer. O Jardim parece esperá-lo, como se cada folha, cada flor, cada partícula de terra soubesse que este momento era inevitável. O ar está pesado com o aroma das memórias, mais denso do que da primeira vez, quase sufocante em sua intensidade.

O jardineiro está lá, cuidando de uma pequena muda que parece ter sido plantada recentemente, suas raízes ainda não completamente estabelecidas, sua existência ainda precária, suspensa entre o ser e o não-ser. Ao ver o Relojoeiro, ele não demonstra surpresa, apenas ergue os olhos cansados e acena como quem reconhece um velho amigo, ou talvez um antigo adversário com quem finalmente fez as pazes.

``Você voltou,'' diz o jardineiro, e não é uma pergunta, mas uma constatação. ``Eu sabia que voltaria.''

``Eu também,'' responde o Relojoeiro, ``embora só agora compreenda que sempre soube.''

A mulher que ele conheceu em sua primeira visita surge entre as árvores, seu rosto uma máscara de eternidade que esconde camadas infinitas de histórias não contadas, de vidas não vividas. Ela se aproxima sem pressa, seus passos tão leves que não perturbam sequer o mais delicado fio de orvalho.

``Nós o esperávamos,'' diz ela, sua voz como o sussurro do tempo passando por folhas secas. ``A árvore o esperava.''

O Relojoeiro segue ambos até sua árvore, aquela mesma que lhe foi mostrada antes, aquela que contém todas as memórias que perdeu, todas as versões dele mesmo que foram esquecidas, abandonadas ao longo do caminho. Mas a árvore está diferente agora. Parece maior, mais robusta, como se na sua ausência tivesse continuado a crescer, alimentada por novas memórias, por novos esquecimentos.

``Durante sua jornada,'' explica o jardineiro, ``você perdeu mais de si mesmo, e tudo isso foi parar aqui. Cada revelação que teve, cada verdade que vislumbrou, cada certeza que abandonou, tudo isso gerou novas folhas, novos ramos, novas raízes que se aprofundaram no solo do tempo.''

O Relojoeiro se aproxima da árvore, toca seu tronco com a reverência de quem toca um altar. Sob seus dedos, a madeira parece pulsar com uma vida própria, um batimento que ressoa com seu próprio coração, como se a separação entre ele e a árvore fosse apenas uma ilusão, uma convenção arbitrária que já não tem mais sentido, que já não tem mais propósito.

``Eu compreendo agora,'' diz o Relojoeiro, e sua voz tem o timbre de quem fala não apenas para os presentes, mas para todos os que vieram antes e todos os que virão depois, para o próprio tecido do tempo que se estende infinitamente em todas as direções. ``Não vim buscar respostas. Vim oferecer-me como resposta.''

Ele retira do pequeno saco de couro os cinco objetos, símbolos das cinco verdades parciais que descobriu em sua jornada, e os dispõe cuidadosamente ao redor da base da árvore, formando um círculo perfeito, uma mandala de significados fragmentados que, juntos, compõem uma imagem ainda incompleta, ainda insuficiente, mas que aponta para algo além de si mesma, para uma totalidade que não pode ser contida em formas finitas, em conceitos limitados, em palavras inadequadas.

``Veja,'' diz ele, apontando para a flor murcha, ``em um mundo onde o tempo flui ao contrário, a beleza está no declínio, não no florescimento. A perfeição não é um estado a ser alcançado, mas uma condição a ser transcendida.''

Toca então o fragmento de rocha, seus dedos mapeando as estrias, as fissuras, as cicatrizes deixadas por uma eternidade comprimida em matéria. ``Em um mundo onde nada muda, a própria imutabilidade se torna uma prisão. A permanência absoluta é tão insuportável quanto a transformação incessante.''

Recolhe a folha de sua árvore, aquela que trouxe consigo em sua primeira visita, agora mais seca, mais frágil, quase translúcida em sua delicadeza. ``Em um mundo onde as memórias são preservadas, descobrimos que o esquecimento é tão necessário quanto a lembrança. É no espaço entre recordar e esquecer que habitamos, que existimos, que somos.''

Ergue o cristal contra a luz, permitindo que seus múltiplos facetas refratem cores que não têm nome, que não podem ser categorizadas, que existem apenas no limiar da percepção. ``Em um mundo de especulações filosóficas, aprendemos que toda verdade é parcial, toda certeza provisória, todo conhecimento limitado. A sabedoria não está em saber, mas em reconhecer os limites do saber.''

Finalmente, toma entre os dedos a engrenagem quebrada, seus dentes torcidos, sua forma distorcida pela violência da revolução. ``E em um mundo de rebelião, compreendemos que destruir não é suficiente. Que qualquer revolução contra o tempo é também uma revolução dentro do tempo, limitada pelas mesmas estruturas que pretende transcender.''

Olha então para o jardineiro e para a mulher, seus rostos agora claramente o mesmo rosto visto de ângulos diferentes, de tempos diferentes, como se fossem não indivíduos distintos, mas manifestações da mesma consciência observando-se a si mesma através do prisma fraturado da temporalidade.

``Eu nasci no tempo,'' diz o Relojoeiro, sua voz agora adquirindo a qualidade sonora de uma declaração final, de um testemunho definitivo, como se cada palavra fosse esculpida não em ar, mas em uma substância mais duradoura, mais essencial. ``Cresci medindo-o, dividindo-o, vendendo-o em parcelas cuidadosamente reguladas. Acreditei ser seu mestre quando era apenas seu servo. Pensei controlá-lo quando era por ele controlado.''

Ele caminha lentamente ao redor da árvore, seus olhos registrando cada detalhe, cada nuance, cada variação de textura e cor, como se estivesse memorizando um rosto amado antes de uma separação que sabe ser final, irrevogável, absoluta.

``Busquei compreender o tempo, dominá-lo, transcendê-lo, e em cada etapa desta jornada, perdi um pouco mais de mim mesmo. Ou talvez, em cada perda, tenha me aproximado um pouco mais do que realmente sou, daquilo que sempre fui antes de me fragmentar em segundos, minutos, horas.''

Sua voz agora adquire uma qualidade metálica, uma ressonância que faz vibrar não apenas o ar, mas o próprio solo sob seus pés, as próprias raízes da árvore que se estendem profundamente na terra, conectando-se com todas as outras árvores, com todas as outras memórias, com todas as outras existências.

``Como Antígona diante de Creonte, recuso-me a obedecer às leis impostas pelos tiranos do tempo linear. Como ela, escolho honrar uma lei mais antiga, mais profunda, mais verdadeira. Não a lei promulgada por deuses ou homens, mas a lei inscrita na própria estrutura do ser, na própria natureza da existência.''

O jardineiro e a mulher permanecem imóveis, testemunhas silenciosas de uma transformação que parece não apenas inevitável, mas necessária, não apenas necessária, mas desejada, não apenas desejada, mas eternamente presente como possibilidade aguardando sua realização, sua manifestação, sua concretização.

``Vós que viveis na ilusão do tempo sequencial, que vos agarrais à segurança ilusória do antes e depois, do causa e efeito, da origem e destino, escutai-me!'', sua voz agora é um trovão, uma tempestade, um cataclismo que sacode os alicerces do jardim, que faz tremer as próprias raízes da realidade. ``Não existe passado nem futuro, apenas o eterno presente que contém em si todas as possibilidades, todas as realizações, todas as existências!''

Os cinco objetos dispostos em círculo começam a vibrar, a ressoar em frequências impossíveis, emitindo sons que não pertencem ao espectro audível, cores que não pertencem ao espectro visível, energias que não podem ser medidas, quantificadas, categorizadas por instrumentos fabricados dentro dos limites do tempo linear.

``Eu, que fui relojoeiro, que medi e dividi o indivisível, que fragmentei o contínuo, que aprisionei o infinito em caixas de segundos, minutos e horas, renuncio a este ofício! Renuncio a esta ilusão! Renuncio a esta tirania que impus a mim mesmo e aos outros!''

E com estas palavras, ele abraça o tronco da árvore, sentindo contra seu peito o pulsar lento, constante, eterno da seiva que flui por suas veias vegetais, da vida que se manifesta não em unidades discretas, em momentos separados, em instantes isolados, mas em um fluxo contínuo, ininterrupto, indivisível de pura existência.

``Aqui permaneço,'' declara, ``não como prisioneiro do tempo, mas como seu libertador. Não como vítima da eternidade, mas como sua testemunha. Não como servo do chronos, mas como manifestação do kairos, do momento oportuno, da plenitude do instante que contém em si toda a eternidade.''

E enquanto fala, algo extraordinário começa a acontecer. A fronteira entre sua carne e a madeira da árvore começa a dissolver-se, não como uma transformação física, mas como uma revelação da verdade que sempre esteve ali, oculta apenas pela ilusão da separação, pela falsa dicotomia entre observador e observado, entre sujeito e objeto, entre ser e tempo.

``Para vós, que ainda habitais o reino do tempo fracionado, parecerá que me tornei parte desta árvore, que abandonei minha humanidade para fundir-me com ela. Mas a verdade é mais simples e mais complexa: sempre fomos um só. Eu e esta árvore, eu e minhas memórias, eu e meus esquecimentos, eu e o tempo, todos aspectos da mesma realidade indivisível que apenas aparenta dividir-se por conveniência, por limitação, por incapacidade de apreender o todo em sua vertiginosa totalidade.''

O jardineiro se aproxima, seus olhos agora claramente os mesmos olhos do Relojoeiro, apenas mais antigos, mais sábios, mais resignados à inevitabilidade do ciclo eterno que não é repetição, mas renovação; que não é retorno, mas reinvenção; que não é círculo fechado, mas espiral ascendente.

``Bem-vindo ao lar,'' diz ele, e sua voz é simultaneamente a voz do jardineiro e a voz do Relojoeiro, a voz do presente e a voz do futuro, a voz do que é e a voz do que será. ``Você finalmente completou o ciclo.''

``Não,'' responde o Relojoeiro, sua voz agora indistinguível do sussurro das folhas movidas por uma brisa que parece surgir do interior da própria árvore, do interior do próprio tempo. ``Não completei o ciclo. Transcendi-o. Não há ciclo a ser completado porque não há separação a ser superada. O início e o fim são ilusões, pontos arbitrários em uma circunferência infinita.''

A mulher também se aproxima, e em seus olhos o Relojoeiro vê todos os que amou, todos os que perdeu, todos os que esqueceu e todos os que o esqueceram, todos unidos não por laços temporais, não por relações causais, mas pela própria natureza da existência que é, em sua essência mais profunda, relação, conexão, interdependência.

``Eles virão,'' diz ela, referindo-se a outros que, como o Relojoeiro, eventualmente encontrarão o caminho até o jardim, até a árvore, até a compreensão que transcende a compreensão. ``Outros Relojoeiros, outros Buscadores, outros Peregrinos do Tempo. E você estará aqui para recebê-los, como parte desta árvore, como parte deste jardim, como parte desta verdade que não pode ser dita, apenas vivida.''

``Sim,'' concorda o Relojoeiro, sua voz agora mais fôlego que som, mais pensamento que palavra, mais silêncio que discurso. ``Estarei aqui, não como eu mesmo, mas como parte de algo maior, algo mais verdadeiro, algo mais real do que qualquer identidade limitada pelo tempo, qualquer consciência confinada à sequencialidade, qualquer existência aprisionada na linearidade.''

E com estas palavras, a fusão se completa, não como um fim, mas como um reconhecimento, uma aceitação, uma celebração da verdade que sempre esteve presente: o Relojoeiro e a árvore, o observador e o observado, o ser e o tempo, todos aspectos da mesma realidade indivisível que apenas se fragmenta no espelho quebrado da percepção limitada, da consciência condicionada, da linguagem inadequada.

Onde antes havia um homem e uma árvore, agora há apenas a árvore, mais robusta, mais viva, mais radiante em sua silenciosa sabedoria, em sua paciente eternidade, em sua generosa inclusividade que acolhe todas as memórias, todos os esquecimentos, todas as versões possíveis e impossíveis do que foi, do que é, do que será e do que poderia ter sido.

\bigskip

E assim, caro leitor, nossa jornada aparentemente chega ao fim, embora saibamos agora que não há fim, apenas transformação; não há conclusão, apenas continuidade; não há resolução, apenas dissolução das falsas dicotomias, das ilusórias separações, das arbitrárias demarcações que impomos à realidade por incapacidade de abraçá-la em sua desconcertante totalidade.

O tempo, essa entidade caprichosa que tentamos aprisionar em relógios, em calendários, em narrativas com começo, meio e fim, revela-se finalmente não como tirano a ser derrubado, nem como enigma a ser decifrado, mas como o próprio meio em que existimos, o próprio tecido de que somos feitos, a própria substância que somos e que nos constitui.

E talvez, caro leitor, você também seja parte desta árvore, parte deste jardim, parte desta história que não é apenas uma história entre muitas, mas a única história que sempre foi e sempre será contada, a história do tempo que se descobre a si mesmo através de nós, que se experimenta a si mesmo através de nossas alegrias e dores, que se conhece a si mesmo através de nossas buscas e descobertas, que se ama a si mesmo através de nossos amores e perdas.

Talvez, ao fechar este livro, você não esteja realmente terminando uma leitura, mas continuando uma conversa iniciada muito antes de seu nascimento e que prosseguirá muito além de sua morte, uma conversa entre todas as consciências que já existiram, que existem e que existirão, unidas não pelo fio frágil da causalidade linear, mas pela teia robusta da simultaneidade eterna, da coexistência atemporal, da interpenetração infinita de todos os seres, todos os momentos, todas as possibilidades.

Ou talvez --- e esta é a possibilidade mais vertiginosa, mais libertadora, mais transformadora --- não haja ``talvez'', não haja ``você'', não haja ``eu'', não haja ``tempo'', apenas a pulsação eterna do ser que se manifesta como multiplicidade aparente, como diversidade ilusória, como separação temporária, apenas para retornar sempre, inevitavelmente, alegremente, à unidade fundamental, à integridade essencial, à totalidade primordial que é nossa verdadeira natureza, nossa autêntica condição, nossa legítima herança.

E assim, sem fim porque nunca realmente começou, sem conclusão porque nunca realmente se desenvolveu, sem despedida porque nunca realmente nos encontramos, nossa história --- que é a única história, que é todas as histórias --- continua, perpetuamente presente, eternamente atual, infinitamente real no único momento que existe: agora.

\vfill

\begin{center}
\copyright{} 2025 Gustavo Mendes e Silva. Todos os direitos reservados.
\end{center}
