\chapter{O Vale}

Voltando ao início mais uma vez, nosso Relojoeiro reabre as cortinas da oficina com um olhar que agora mistura a curiosidade de um explorador e o cansaço de um velho marinheiro, ou talvez não seja realmente cansaço o que sente, mas aquela peculiar forma de lucidez que chega quando todas as ilusões começam a desmoronar, quando os alicerces que sustentavam sua compreensão do mundo revelam-se tão frágeis quanto casas de papel em uma tempestade, é mais um despojamento, uma nudez essencial que o deixa vulnerável às verdades que sempre estiveram ali, escondidas sob o véu da cotidianidade, da rotina, da certeza não examinada, e é com essa nova vulnerabilidade, esse novo estado de receptividade quase dolorosa, que ele decide seguir adiante, não para trás em direção à segurança ilusória das antigas crenças, não para os lados em tentativas desesperadas de evitar o inevitável, mas para um destino que promete, se não respostas, ao menos perguntas mais precisas, mais essenciais, e não é isso, afinal, o início de toda verdadeira sabedoria? não questionar mais, mas questionar melhor?

Fechando a oficina, ele guarda a chave no mesmo bolso onde mantém a carta misteriosa e a flor murcha da Cidade das Horas Invertidas, objetos que agora funcionam como talismãs ou âncoras em um mundo que se torna cada vez mais fluido, cada vez mais resistente às categorizações fáceis, aos nomes convencionais, às explicações consagradas que os seres humanos inventam não para compreender o universo, mas para protegerem-se da vertigem de sua incompreensibilidade fundamental, pequenas mentiras consensuais que permitem seguir adiante sem confrontar o abismo que existe sob cada passo, sob cada pensamento, sob cada respiração, o Relojoeiro sente o peso desses objetos contra seu corpo como um lembrete constante do que deixou para trás e do que ainda busca, são pontos de contato com o real que o impedem de flutuar completamente para fora de si mesmo, para longe de tudo o que um dia considerou verdadeiro, e há uma ironia perversa nisso, pensa ele com um sorriso que ninguém vê, que uma flor morta colhida em uma cidade impossível seja agora um de seus poucos vínculos com a realidade, uma prova tangível de que o que experimenta não é apenas um sonho elaborado, uma alucinação coerente, um delírio sistemático produzido por uma mente saturada de tempo, intoxicada por segundos, minutos, horas.

O caminho que se desenrola à sua frente não consta em nenhum mapa, não respeita nenhuma coordenada geográfica conhecida, parece existir apenas como uma função de sua própria necessidade de percorrê-lo, uma manifestação física de um impulso interno, como se a estrada fosse o prolongamento visível de seus questionamentos mais profundos, de suas dúvidas mais essenciais, e não deveria surpreendê-lo, pois se o tempo já não é o que parecia ser, por que o espaço manteria sua natureza convencional, sua obediência às leis da física? por que qualquer aspecto da realidade permaneceria intacto quando sua componente mais fundamental, aquela que permeia e sustenta todas as outras, é revelada como algo completamente diferente do que sempre supusemos? aquele que questiona o tempo deve estar preparado para que todas as outras certezas desmoronem em sequência, como dominós alinhados com precisão maníaca, pois nenhuma verdade existe isolada, nenhum conceito sobrevive à reformulação completa de seu contexto, e o contexto último de toda experiência humana não é o espaço, não é a matéria, não é a energia, mas o tempo, esse contínuo inescapável que nos atravessa, nos constitui, nos delimita.

À medida que avança, observa que a paisagem se torna cada vez mais estática, como se um glacialista invisível houvesse reduzido a temperatura do mundo até quase o zero absoluto, congelando o fluxo natural das coisas, suspendendo os processos mais básicos, árvores que parecem esculpidas em uma substância que imita a vida sem possuí-la realmente, folhas que não tremem mesmo quando o vento sopra (mas haverá realmente vento aqui?), riachos cujas águas são tão imóveis que mais parecem vidro polido, reflexos perfeitos que nunca se distorcem, que nunca vibram com aquela imprecisão característica de tudo que é vivo, e o Relojoeiro se pergunta se ainda está no mesmo mundo que conheceu, ou se atravessou sem perceber alguma fronteira invisível para um território onde as leis fundamentais da física, da biologia, do tempo operem segundo princípios completamente diferentes, ou não operem de forma alguma.

Quando finalmente avista o Vale da Imutabilidade, o nome surge espontaneamente em sua consciência como se tivesse sido sempre óbvio, como se não pudesse existir outro título para este lugar, este não-lugar, esta suspensão da lógica convencional em favor de uma outra lógica que ainda não compreende completamente mas pode intuir, e há nessa intuição um elemento de terror ancestral, como se alguma parte primitiva de sua mente reconhecesse um perigo mais fundamental do que qualquer ameaça física, um perigo para a própria coerência de sua identidade, de sua existência como ser no tempo, o vale se estende abaixo dele, uma depressão geográfica perfeitamente simétrica, como se tivesse sido esculpida por um artista obsessivo e não pelos processos caóticos da erosão, do movimento tectônico, das forças cegas que normalmente moldam as paisagens, e no fundo do vale, dispostas com uma ordem que sugere intenção, planejamento, vontade consciente, as habitações dos que escolheram (ou foram escolhidos para) viver na imutabilidade.

O primeiro encontro é tão repentino quanto inexplicável, pois não viu nem ouviu ninguém se aproximar, não houve passos, respiração, o ruído característico de tecidos em movimento, nenhum dos pequenos sinais que anunciam a presença humana, apenas a súbita materialização de um homem à sua frente, ou talvez não tenha sido súbita, talvez tenha sido gradual e o Relojoeiro, habituado ao ritmo acelerado do mundo convencional, simplesmente não tenha notado o lento surgimento dessa figura alta, ereta, impassível como uma estátua grega, talvez o homem sempre tenha estado ali e foi o Relojoeiro quem, por razões que nem mesmo ele compreende, apenas agora se tornou capaz de percebê-lo, de registrar sua presença, e há nessa possibilidade algo tão perturbador quanto na primeira hipótese, pois sugere não apenas um mundo diferente do que conhece, mas um mundo fundamentalmente inapreensível, onde nem mesmo seus sentidos, essas janelas primárias para a realidade, são confiáveis.

``Bem-vindo ao Vale da Imutabilidade,'' diz uma voz grave e profunda, que parece emanar não apenas do homem à sua frente, mas das próprias pedras, do próprio ar, como se todo o ambiente fosse uma extensão daquele ser, ou como se aquele ser fosse uma manifestação localizada, uma concentração do ambiente, ``aqui, o tempo é um conceito ultrapassado, um luxo que decidimos dispensar'', e o homem não se move enquanto fala, seu rosto permanece tão estático quanto as rochas, seus olhos fixos, quase vítreos, e no entanto transmitindo um tipo de inteligência, de consciência, que o Relojoeiro reconhece imediatamente como não-humana, não porque seja inferior à humanidade, mas porque a transcende, porque existe segundo padrões que nossas mentes, moldadas por milênios de evolução em um ambiente temporal específico, não foram equipadas para compreender, ``por que correr, por que mudar, quando se pode simplesmente existir?'', e há nessa pergunta uma provocação, um desafio à própria essência do que o Relojoeiro sempre considerou valioso, essencial, verdadeiro.

``Como vocês conseguem viver assim, presos em um eterno agora?'', pergunta ele, com seu habitual sarcasmo que agora funciona como uma frágil proteção contra o desconforto profundo que sente diante dessa negação radical de tudo o que define a experiência humana, pois se não há tempo, não há história, nem memória, nem expectativa, nem narrativa, nem aquela tensão fundamental entre o que fomos e o que seremos que constitui a própria textura da identidade pessoal, se não há tempo não há transformação, não há possibilidade, não há escolha, não há liberdade, não há propósito, não há sentido no sentido humano do termo, e como pode algo que não muda ser considerado vivo? não é a capacidade de adaptação, de transformação, de aprendizado, de crescimento, a própria definição do que distingue a vida da não-vida? não é o movimento no tempo a condição sine qua non da existência consciente?

O homem sorri, um sorriso que mais parece um esgar, um arranjo de músculos faciais que imita o que nos seres humanos seria uma expressão de satisfação, de alegria, de compreensão, mas que nele transmite apenas uma espécie de condescendência, de superioridade, de distanciamento irônico, ``a mudança é uma ilusão, uma necessidade fabricada por mentes inquietas, aqui, encontramos paz na constância, na permanência, a dor não nos toca, a alegria não nos move, apenas somos'', e enquanto fala, o homem permanece perfeitamente imóvel, nem mesmo seus lábios se movem, como se a voz fosse uma propriedade do ambiente e não do indivíduo, como se o ar vibrasse espontaneamente em padrões que o cérebro do Relojoeiro interpreta como palavras, como linguagem, como comunicação, embora em um nível mais profundo ele suspeite que o que ocorre não seja realmente comunicação, mas algo completamente diferente, algo para o qual não temos nome nem conceito.

O Relojoeiro observa os habitantes do vale, e percebe que todos compartilham aquela mesma qualidade inquietante, aquela mesma negação do movimento natural, do fluxo esperado, eles se movem, sim, realizam ações, interagem com objetos e entre si, mas seus movimentos parecem ensaiados, predeterminados, executados com a precisão mecânica de autômatos, de marionetes manipuladas por uma consciência central que poderia ser um deus, um programa de computador, um campo de força, um consenso coletivo, um algoritmo, ou nada disso, algo tão distante de nossas categorias que sequer podemos começar a conceituá-lo sem distorcê-lo completamente, há uma perfeição em cada gesto que o torna, paradoxalmente, menos real, como se a própria ausência de erros, de hesitações, de desvios, fosse a prova mais contundente de sua artificialidade, de sua não-humanidade, é como assistir a uma peça onde os atores estão tão comprometidos com seus papéis, tão imersos em suas performances, que perderam completamente o contato com a espontaneidade, com a imprevisibilidade, com a vulnerabilidade que caracteriza a vida fora do palco.

Entre os habitantes, no entanto, uma mulher se destaca, e o Relojoeiro a identifica imediatamente, como se já soubesse que a encontraria ali, como se sua presença fosse uma necessidade narrativa, um ponto de inflexão em uma história que ele não está apenas testemunhando, mas de alguma forma criando à medida que avança, seus olhos, ao contrário dos outros, brilham com uma luz de inquietação, como pequenas fissuras no tecido estático da imutabilidade, pequenas insurreições contra o regime de permanência que governa o vale, há nela uma humanidade que parece ter sido drenada dos demais, uma conexão preservada com o fluxo, com a incerteza, com a busca que caracteriza os seres que existem no tempo e não fora dele, ela se aproxima do Relojoeiro e, com uma voz que parece um sussurro carregado pelo vento, tão diferente da voz inumana do primeiro homem, diz: ``aqui, tudo é perfeito, mas eu sinto falta do imperfeito. A imutabilidade é uma prisão, e eu anseio por algo mais'', e nessas palavras simples, nessa confissão de insuficiência, de incompletude, há mais sinceridade, mais vulnerabilidade, mais humanidade do que em todos os discursos polidos, precisos, incontestáveis dos demais habitantes do vale.

``E por que não deixa este lugar?'', pergunta o Relojoeiro, sabendo que a resposta não será simples, que nada nesta jornada tem sido ou será simples, que a complexidade, a ambiguidade, a contradição são as únicas constantes em um universo onde todas as outras certezas são reveladas como provisórias, como contingentes, como acidentais, ele formulou a pergunta sabendo que ela é simultaneamente óbvia e impossível, pois quem, tendo conhecido a imutabilidade, poderia simplesmente afastar-se dela sem ser transformado, sem ser marcado de forma indelével por essa experiência? quem, tendo contemplado a eternidade, poderia retornar ao tempo sem carregar consigo para sempre a nostalgia do infinito, a memória da perfeição, a cicatriz da totalidade?

``A saída não é permitida'', responde ela, e em sua voz há uma resignação que o Relojoeiro reconhece como exclusivamente humana, esse reconhecimento da própria impotência diante de forças maiores, esse confronto com limitações que não podem ser superadas pela vontade, pela inteligência, pelo esforço, essa aceitação relutante do que não pode ser mudado que constitui, paradoxalmente, o início de toda possível liberdade, ``o Vale da Imutabilidade não tolera deserções. Quem tenta partir é condenado ao esquecimento'', e o Relojoeiro compreende que o esquecimento, neste contexto, não é apenas a perda da memória, da identidade, da história pessoal, mas algo muito mais radical, muito mais absoluto, uma exclusão ontológica, um apagamento não apenas da existência, mas da própria possibilidade de existir, como se o desertor fosse removido não apenas do presente, mas de todo o passado e futuro, como se nunca tivesse nascido, como se nem mesmo a possibilidade de seu nascimento tivesse existido.

A decisão de ajudá-la não é consciente, não é calculada, não é resultado de uma deliberação racional, de uma análise de custos e benefícios, de uma ponderação ética sobre o certo e o errado, é um impulso tão inevitável quanto a queda de uma pedra liberada no ar, tão natural quanto a expansão do universo, tão inquestionável quanto o próprio fluxo do tempo que ele começou a questionar, o Relojoeiro sabe, com uma certeza que transcende a razão, que não pode abandoná-la neste lugar onde o ``ser'' foi dissociado do ``tornar-se'', neste reino de perfeição estéril onde nada morre porque nada verdadeiramente vive, neste simulacro de existência onde a ordem absoluta revelou-se como a mais sutil forma de caos, como a mais completa negação da possibilidade, da criatividade, da novidade que torna a vida não apenas suportável mas desejável, não apenas desejável mas necessária.

Trama uma fuga, um desafio à ordem imutável do vale, um ato de rebeldia cósmica que sabe, desde o início, estar condenado ao fracasso, pois como pode uma criatura do tempo, um ser definido pela finitude, pela temporalidade, pela mudança, desafiar a própria negação desses princípios? como pode o mortal confrontar o eterno em seu próprio território, usando as armas do contingente contra o necessário, do efêmero contra o permanente? e no entanto, essa consciência da futilidade não diminui sua determinação, pelo contrário, a alimenta, a intensifica, porque há uma dignidade essencial na tentativa, no esforço, na recusa em aceitar o inaceitável, mesmo quando ou especialmente quando o fracasso é a única conclusão possível, há uma nobreza no desespero que reconhece sua própria condição sem ser definido por ela, que age apesar da certeza da derrota, que afirma a liberdade precisamente quando ela parece mais ilusória.

Observa os ciclos repetitivos, as rotinas imutáveis, os padrões ossificados que governam a existência no vale, e começa a perceber que a imutabilidade não é realmente uma ausência de tempo, mas um tempo preso em um loop perfeito, um eterno retorno não no sentido nietzschiano de uma repetição cósmica onde cada evento, cada experiência, cada pensamento recorre infinitamente, mas no sentido muito mais restrito, muito mais empobrecido, de um único momento, de uma única configuração do possível que se repete sem variação, sem evolução, sem propósito além da própria perpetuação, é o tempo reduzido à sua expressão mais básica, mais primitiva, mais esquemática, despojado de tudo o que o torna significativo, que o torna fértil, que o torna humano.

É precisamente essa repetição que revela a possibilidade, a única possibilidade, de escape, pois o que se repete com perfeição absoluta torna-se, paradoxalmente, previsível, e o que é previsível pode ser antecipado, e o que pode ser antecipado pode, talvez, ser subvertido, não pela força, não pela velocidade, não pela astúcia, mas pela introdução deliberada de uma variável no sistema perfeito, de uma perturbação no campo de força estático, de um ruído na sinfonia monótona da igualdade, e essa variável, essa perturbação, esse ruído tomará a forma de algo tão simples, tão básico, tão fundamental que os arquitetos da imutabilidade, em sua obsessão pela complexidade, pela totalidade, pela perfeição, talvez o tenham negligenciado completamente: o acaso, a contingência, o imprevisto, não como conceitos abstratos, mas como realidades concretas, como potências do ser.

Na calada da noite, embora ``noite'' seja apenas uma convenção nestas coordenadas onde a luz parece ter uma qualidade constante, imutável, sem as variações circadianas que marcam o ritmo da vida na superfície terrestre, o Relojoeiro e a mulher começam a correr, não como quem foge, mas como quem cria, como quem gera a própria possibilidade de fuga pelo ato mesmo de fugir, cada passo é uma batalha contra a inércia, contra a entropia negativa que caracteriza este lugar onde a ordem não emerge do caos, mas o precede, o determina, o impossibilita, cada respiração é uma afronta à imutabilidade, cada batimento cardíaco uma pequena revolução contra a tirania da permanência, e há algo de glorioso nessa rebelião condenada, nessa insurreição impossível, nesse gesto quixotesco contra moinhos de vento que são, na verdade, deuses disfarçados, leis cósmicas mascaradas de fenômenos meteorológicos.

Conseguem sair do vale, e o Relojoeiro sabe que não deveriam ter sido capazes de fazê-lo, que sua fuga bem-sucedida é uma contradição, uma impossibilidade lógica, uma quebra na estrutura ficcional que sustenta este universo de regras implacáveis, de leis invioláveis, de necessidades absolutas, e no entanto, aqui estão, ofegantes, exaustos, vivos de uma forma que não eram, que não poderiam ser, dentro dos domínios da imutabilidade, mas não sem consequências, pois o Vale, ofendido, parece retaliar, o tempo ao redor deles distorce-se, acelera e desacelera em um frenesi caótico, como uma substância elástica que, tendo sido esticada além de seus limites naturais, agora contrai-se violentamente, indiscriminadamente, arrastando consigo tudo o que encontra em seu caminho, o Relojoeiro, com seu humor ácido, pensa que o tempo é mais caprichoso do que uma criança mimada, mais vingativo do que um amante traído, mais imprevisível do que um bêbado com as chaves de um carro esportivo.

Finalmente, encontram-se fora do vale, em uma terra onde o tempo volta a fluir, mas de forma errática, instável, como se ainda estivesse se recuperando do trauma de ter sido negado, suprimido, aprisionado, a mulher, livre pela primeira vez, sorri com uma alegria que o Relojoeiro jamais havia visto, uma alegria tão pura, tão fundamental, tão absoluta que dói contemplá-la diretamente, como dói olhar para o sol sem proteção, ``obrigada,'' diz ela, ``você me devolveu a vida, o verdadeiro tempo'', e o Relojoeiro compreende, com uma clareza que só vem nos momentos de extrema lucidez, de extrema receptividade, que vivemos não para acumular experiências, não para colecionar realizações, não para construir legados, mas para estes raros, preciosos, irreplicáveis instantes de autenticidade absoluta, estes lampejos de significado que iluminam, por um breve momento, a noite escura do absurdo, da contingência, da finitude que é a condição humana.

\bigskip

E assim, caro leitor, voltamos ao início, ao ponto onde o Relojoeiro, agora com uma nova compreensão e uma nova companheira, reabre as cortinas de sua oficina. Mas algo mudou. A flor murcha, um símbolo da cidade invertida, agora está ao lado de um fragmento de rocha do vale, um lembrete constante de que o tempo é uma dança entre mudança e constância, entre o movimento e a imobilidade, e não seria essa, afinal, a natureza última da realidade? não essa ou aquela polaridade, não esse ou aquele extremo, mas a tensão dinâmica entre opostos, o diálogo perpétuo entre forças contrárias, a síntese contínua que nunca se resolve completamente, que nunca alcança um estado final, que nunca chega a uma conclusão definitiva, porque o momento em que a dialética se completa, em que todas as contradições são resolvidas, em que todas as perguntas encontram suas respostas, é o momento em que o tempo para, em que a história termina, em que a vida, como a conhecemos e a valorizamos, simplesmente deixa de ser?

Era uma manhã como outra qualquer, ou assim parecia ao Relojoeiro, que, ao abrir as cortinas de sua oficina, deixou que os primeiros raios de sol invadissem o espaço repleto de relógios e sombras. Mas algo dentro dele havia mudado, uma inquietação, um formigamento de pensamentos que não se aquietavam, que não se contentavam com o tic-tac monótono dos ponteiros. E foi então que ele decidiu, com a mesma certeza com que se sabe que o dia segue a noite, que era hora de buscar respostas, era hora de entender o tempo, não como um conjunto de engrenagens e mostradores, mas como a essência que permeia tudo, que define a vida e a morte, o amor e a perda, o encontro e a despedida.

E assim, caro leitor, continuamos nossa jornada, sabendo que cada volta ao início é uma nova camada de compreensão, uma nova peça no quebra-cabeça infinito que é o tempo. Prepare-se, pois o próximo capítulo promete mais mistérios, mais revelações e, claro, mais voltas ao começo. Afinal, o tempo, essa entidade caprichosa, adora um bom enigma, e nós, criaturas que somos do tempo, não podemos deixar de jogar seu jogo, mesmo sabendo que as regras mudam constantemente, que o tabuleiro se reconfigura a cada movimento, que a vitória, se é que existe, consiste não em vencer, mas em continuar jogando, em continuar questionando, em continuar buscando aquela verdade que sempre escapa, que sempre se esconde, que sempre nos convida a avançar mais um passo, virar mais uma página, abrir mais uma porta, mesmo que saibamos, no fundo, que do outro lado encontraremos apenas mais perguntas, mais enigmas, mais labirintos, porque talvez seja esse, afinal, o propósito último da existência: não encontrar, mas buscar; não responder, mas perguntar; não chegar, mas caminhar eternamente em direção a um horizonte que se afasta na mesma medida em que nos aproximamos dele.

\begin{center}
\textit{O fragmento de rocha do Vale da Imutabilidade}
\end{center}
