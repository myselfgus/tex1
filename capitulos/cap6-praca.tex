\chapter{A Praça}

Voltando ao início mais uma vez, o Relojoeiro reabre as cortinas da oficina, agora com uma determinação quase fanática, travessia permanente, terceiras margens cruzando, como se estivesse prestes a enfrentar o próprio Tempo em um duelo de morte --- \textit{nonada!} ---, os objetos têm sombras que não lhes pertencem e o ar circula como uma guilhotina invisível entre passado e futuro, ele sente que cada volta ao começo o empurra mais fundo em espiral de descobertas e perplexidades, atravessando, ele atravessa, todas as coisas atravessa, cada retorno diferente do anterior como rio que nunca é o mesmo, nem para o rio nem para quem nele mergulha duas vezes, o Tempo é Sertão, vasto, traiçoeiro, sem começo definido nem fim verdadeiro, território onde Deus e o Diabo se encontram para negociar as horas. Fechando a oficina novamente, trancafio de mundos, ele segue em direção ao próximo destino: a Revolução dos Ponteiros, um lugar onde o tempo é questionado, desafiado e, finalmente, subvertido, como criança que desmonta brinquedo para entender seu funcionamento e o destrói exatamente por isso.

A cidade se descortina aos poucos, fragmentada, objetos cortados da realidade com tesouras cegas, cacos de espelho refletindo pedaços de céu que nunca existiram, árvores feitas de fumaça, nuvens feitas de pedra, as ruas se contorcem como víboras feridas, edifícios inclinados para todos os lados exceto para cima, e o Relojoeiro olho-que-tudo-vê e nada compreende, o Relojoeiro sente um estranhamento visceral, um desenraizamento da própria certeza de existir, pois até mesmo a ordem espacial habitual foi deformada, distorcida, virada do avesso e costurada novamente com linhas feitas de intervalos, a cidade é menos um lugar e mais uma pergunta, os tijolos das casas são interrogações empilhadas, as janelas são reticências abertas para um céu que talvez seja apenas o vazio tingido de azul, a Revolução, ele descobre, não teve início em um momento específico --- \textit{nonada que era!} --- sempre esteve acontecendo, espiral que se fecha e se abre simultaneamente, cobra engolindo a própria cauda enquanto continua crescendo.

Ao se aproximar do local, o Relojoeiro nota uma agitação que contrasta violentamente com a calma estática do Vale da Imutabilidade, lugar-não-lugar ficado para trás, mundo-sumiço nas veredas das possibilidades, aqui o tempo não é apenas desafiado, mas atacado com fúria revolucionária, marretadas no vidro protetor dos relógios de parede, fogueiras alimentadas com calendários, ampulhetas quebradas derramando areia que, em vez de cair, sobe como poeira rebelde desafiando a gravidade, os habitantes, homens e mulheres em cujos rostos o Relojoeiro vê simultaneamente juventude e decrepitude, inocência e sabedoria, como se suas identidades fossem colagens malfeitas, retalhos de épocas diversas costurados à força em um único semblante, têm olhares intensos e gestos decididos, preparando-se para uma revolução que promete libertar o tempo de suas correntes e, quem sabe, libertar a eles mesmos, da sina de esperar, de transformar a vida em espera, cada segundo consumindo o próximo como num rosário de contas invisíveis.

``Tempo não corre mais, parado está, esperando que o libertemos dele mesmo'' --- voz da mulher cortava a tarde como faca corta fruta madura --- ``Vamos tomar a torre do relógio! Vamos quebrar as correntes do tempo e mostrar ao mundo que somos donos do nosso próprio destino!'' --- crepitação de folha seca esmagada, futuro pisoteado antes de nascer. A multidão, movendo-se como criatura única com mil cabeças, avança em direção à torre central que se ergue na praça, um obelisco monstruoso cujo relógio gigante projeta sombra de prisão circular sobre as ruas, sobre os rostos, sobre as almas de todos os que vivem sob sua autoridade, sob seu tique-taque perpétuo que é grilhão invisível, respirador artificial de uma sociedade moribunda que respira apenas porque o relógio lhe concede essa permissão, um segundo de cada vez, um minuto de cada vez, uma hora de cada vez, nada mais, nada além, apenas a medida exata de vida necessária para continuar funcionando, para continuar servindo, para continuar existindo sem jamais realmente viver.

``Bem-vinda ao coração da revolta,'' diz uma jovem de olhos flamejantes e voz apaixonada, colares feitos de engrenagens quebradas, pulseiras de cordas de relógio arrebentadas, o cabelo preso com pequenos ponteiros, como se houvesse incorporado à própria aparência os despojos de seus inimigos vencidos, ``aqui não aceitamos o tempo como um tirano. Estamos prontos para quebrar os ponteiros e redefinir nossas vidas.'' A jovem estende a mão, um convite, uma iniciação, um pacto de sangue com o futuro incerto que pretendem criar, e em sua palma o Relojoeiro vê linhas que não formam os padrões habituais, não há linha da vida, linha do coração, linha da cabeça, apenas espirais concêntricas que parecem pulsar como se tivessem vida própria, como se fossem pequenos vórtices tentando sugar quem os observa para uma dimensão desconhecida, para um território além do mapeável, além do concebível, além do tempo.

O homem que foi o Relojoeiro que talvez já não o seja mais, pois as identidades se dissolvem nesta cidade como açúcar em água quente, com sua habitual ironia que é escudo e lança, armadura e ferida, pensa em responder que talvez eles estivessem confundindo libertação com anarquia, que talvez uma revolução contra o tempo seja tão impossível quanto uma revolta contra a gravidade, contra a morte, contra a própria condição de ser finito em um universo onde apenas a finitude parece ter alguma realidade tangível, mas contém-se, engole as palavras que teriam saído como arame farpado de sua garganta, em vez disso, ele observa, olho-tempo que tudo observa para depois relatar a estranhos que não existem, pois às vezes observar é a melhor maneira de entender o caos antes de mergulhar nele, a cabeça cheia de sussurros de outros lugares visitados antes, fragmentos de certezas que se dissolvem, tesouras do regime cortando as últimas fotos de família, o passado perseguido pelos cães da censura.

``Vamos tomar a torre do relógio!'' grita um líder carismático, cuja voz reverbera pela praça central como trovão num céu sem nuvens, uma contradição materializada, uma impossibilidade tornada real pela pura força de vontade, pela pura intensidade de propósito, pela pura convicção de que o impossível é apenas o possível que ainda não encontrou ocasião de manifestar-se, ``Vamos quebrar as correntes do tempo e mostrar ao mundo que somos donos do nosso próprio destino!'', e sua voz é semente em terra fértil, é faísca em palha seca, é pedra na superfície de lago plácido, gerando ondas concêntricas que se propagam, ampliam, multiplicam, até que não há um único coração na praça que não pulse no mesmo ritmo, não há um único peito que não respire a mesma indignação, não há uma única alma que não abrigue o mesmo sonho de liberdade.

O Relojoeiro se vê arrastado pela multidão como folha em redemoinho, verá-tinha que segue o fluxo universal, sentindo-se como um turista em uma guerra civil, um observador passivo em uma revolução que não compreende totalmente, as pessoas ao seu redor se movem com uma determinação alimentada pela raiva, pelo desespero, pela esperança, pela possibilidade de um amanhã diferente, um amanhã que não seja apenas mais um dia idêntico aos anteriores, marcado pela mesma sucessão monótona de horas, pela mesma progressão impiedosa de segundos que devoram a vida sem oferecer nada em troca, nem significado, nem propósito, nem consolação, vemo-nos naquilo que nos olha, olho de vida olhando olho da morte, espelho partido refletindo o rosto da transgressão --- \textit{``eu queria e não queria, ah, mas queria sim''} ---, sabe que deveria tentar escapar, retornar à segurança relativa de sua condição de observador, mas há algo nessa luta que ressoa dentro dele, algo que desperta uma chama adormecida, uma brasa esquecida sob as cinzas frias da rotina, da aceitação, da resignação que por tanto tempo confundiu com sabedoria.

A multidão avança como uma maré irresistível, humana-criatura com membros feitos de indignação e torsos feitos de esperança, e o Relojoeiro, agora parte dela, parte absorvida pelo todo sem perder completamente a individualidade, como gota de chuva que se junta ao oceano sem deixar de ser água, se encontra diante da torre do relógio, um monstro de pedra que parece zombar de seus esforços, com suas paredes maciças, suas janelas estreitas como olhos semicerrados de um gigante observando com desdém os pequenos seres que ousam desafiá-lo, seu relógio no topo, tão alto que seria necessário torcer o pescoço até o limite para enxergá-lo completamente, cada número romano uma sentença, cada movimento dos ponteiros um decreto, cada tique-taque uma chicotada nas costas do tempo subjugado que, apesar de ser a própria substância da existência, foi aprisionado, domesticado, escravizado por sua própria criação, por sua própria manifestação, por sua própria prole.

``Derrubem-no!'' grita o líder, e com uma determinação feroz, os revolucionários atacam o símbolo do tempo tirânico, alguns com martelos, outros com picaretas, alguns com as próprias mãos sangrando, dilacerando-se contra a pedra impassível que parece absorver a violência, a fúria, o desespero sem mostrar o menor sinal de fraqueza, de dano, de cedência. O Relojoeiro, o homem que já se reconhece nos outros como se todos fossem um, intrépido buscador de respostas, empunhando uma barra de ferro que encontrou no chão, herança retorcida de alguma construção anterior, de algum projeto abandonado, de algum sonho interrompido, golpeia o relógio com toda a força de suas convicções recém-descobertas, com todo o peso de suas dúvidas antigas, com toda a energia de sua busca interminável por uma verdade que talvez seja impossível de alcançar, que talvez não exista de forma alguma, que talvez seja apenas o horizonte sempre distante que nos mantém em movimento, que nos impede de desistir, de aceitar, de conformar-nos com as respostas fáceis, com as verdades convenientes, com as certezas confortáveis que são o ópio dos que temem o desconhecido.

Cada golpe é uma afirmação, um grito de liberdade, um desafio ao inevitável tic-tac que dita suas vidas --- batendo, pancada, pedaço por pedaço, lá-é-aqui que atravessa os vivos e os mortos ---, o pisar pesado dos revolucionários fazendo estremecer o chão, a vibração subindo pelas paredes como uma premonição sísmica, como um aviso de que algo fundamental está prestes a mudar, a romper-se, a transformar-se em algo diferente, algo novo, algo ainda sem nome porque o vocabulário do futuro ainda não foi inventado, ainda não foi sonhado, ainda não foi sequer imaginado como possibilidade concreta. \textit{``Assim falei, assim é''} --- a realidade dobra-se quando se ousa falar o que ainda não existe, nomeação é criação, verbo é fertilização da vastidão do nada.

Os golpes, primeiro descoordenados, caóticos, individuais, começam a sincronizar-se, a unificar-se em um único ritmo, um único impacto, uma única vontade manifestada através de centenas de braços, de centenas de ferramentas, de centenas de intenções que, fundidas, tornam-se uma força capaz de alterar a própria estrutura da realidade, de perfurar a membrana do possível para alcançar o impossível que espera do outro lado, ansioso por nascer, por materializar-se, por existir além da mera conceitualização, além da mera potencialidade, além da mera especulação teórica. \textit{``É neste tempo-sem-tempo que existimos verdadeiramente''}, diz uma voz, ou talvez um pensamento, ou talvez um eco do futuro viajando de volta para testemunhar seu próprio nascimento.

Uma primeira fissura aparece na base da torre, tênue como um fio de cabelo, insignificante como o primeiro momento de dúvida em uma fé até então inabalável, desprezível como a primeira gota de água que escapa da represa aparentemente impenetrável, mas ela está lá, visível para quem sabe olhar, para quem aprendeu que as grandes transformações começam sempre nos detalhes mais ínfimos, nas alterações mais sutis, nas mudanças quase imperceptíveis que, num piscar de olhos, num suspiro do universo, num segundo distraído do destino, tornam-se avalanches, inundações, revoluções que redefinem o curso da história, o fluxo do tempo, a natureza mesma da existência. \textit{``O grande é pequeno e o pequeno é grande, depende de onde se vê, vendo por onde''} --- fala o Relojoeiro para quem não ouve, palavras apenas para si mesmo, única audiência verdadeira de todos os nossos pensamentos.

A fissura cresce, ramifica-se como veias num corpo vivo, estende-se para cima, para os lados, explorando as fragilidades ocultas da estrutura, os pontos onde a pedra é menos densa, menos resistente, menos capaz de sustentar o peso não apenas da própria torre, mas de tudo o que ela representa, de todo o sistema que ela sustenta, de toda a ordem que ela impõe sobre um mundo que, em sua natureza mais básica, mais fundamental, mais essencial, talvez não deva ser ordenado, talvez não possa ser contido, talvez resista intrinsecamente a qualquer tentativa de domesticação, de categorização, de medição. O relógio no topo começa a titubear, os ponteiros tremem como agulhas de bússola próximas a um campo magnético potente, oscilando entre o norte verdadeiro e o norte distorcido, entre o tempo convencional e o tempo revolucionário, entre o chronos que devora seus filhos e o kairos que oferece redenção.

Enquanto o relógio desmorona, engrenagem por engrenagem, parafuso por parafuso, peça por peça, vidro estilhaçado refletindo centenas de futuros possíveis em cada fragmento, algo dentro do Relojoeiro também se quebra, fissuras dividindo o que antes fora sólido, o Olho transforma em palavras, as palavras em objetos, os objetos voltam ao vazio --- milagre ao contrário, tempo redimido de sua própria tirania ---, se quebra também no revolucionário que foi por alguns instantes, quando incorporou a fúria coletiva, a indignação compartilhada, a esperança comum de uma nova ordem, uma nova lógica, uma nova relação com o tempo e, por extensão, com a própria vida. Ele percebe que a revolução não é apenas contra o tempo, mas contra as limitações que impõem a si mesmos, contra as fronteiras artificiais que separam o possível do impossível, contra a divisão arbitrária entre o que é e o que poderia ser, se tivéssemos a coragem, a visão, a determinação de transcender as categorias herdadas, as definições recebidas, as estruturas impostas que confinam tanto o pensamento quanto a experiência em corredores estreitos, em caminhos predeterminados, em rotas aprovadas que levam apenas aos destinos sancionados, nunca às paisagens inexploradas onde novas formas de existência nos aguardam.

Em meio aos destroços, ele encontra o líder carismático, agora olhando para o vazio com um misto de triunfo e desespero, de realização e perplexidade, de vitória e desorientação, os olhos fixos no espaço antes ocupado pela torre, na ausência que agora é mais significativa, mais poderosa, mais transformadora do que qualquer presença poderia ser, o silêncio do relógio destruído mais eloquente do que seu anterior tique-taque perpétuo, o vazio deixado pela estrutura mais substancial do que a própria estrutura jamais foi em sua materialidade limitada, em sua existência circunscrita, em sua realidade finita.

``Conseguimos,'' diz o líder, falante que fala para fazer real o sonho recém-nascido, mágico que pronuncia encantamento para concretizar a ilusão, criança que nomeia o invisível para torná-lo palpável, mas sua voz não tem a certeza de antes, o timbre vacila entre a afirmação e a interrogação, a entonação oscila entre a declaração e o questionamento, o volume flutua entre o anúncio e o sussurro, como se a vitória tão longamente sonhada, tão ardentemente desejada, tão ferrenhamente perseguida, agora que finalmente foi alcançada, revelasse sua natureza ambígua, sua essência contraditória, sua qualidade paradoxal de ser simultaneamente culminação e começo, conclusão e introdução, resposta e pergunta. ``E agora, o que faremos?'', e esta é a verdadeira questão revolucionária, a única que realmente importa, a única que contém em si o germe de todas as outras perguntas, de todas as outras dúvidas, de todas as outras possibilidades, pois qualquer revolução que saiba exatamente o que fazer após a vitória não é revolução verdadeira, é apenas substituição, apenas troca, apenas permutação de uma ordem por outra, de uma estrutura por outra, de uma limitação por outra.

O Relojoeiro, com a sabedoria irônica de quem já viu demais, de quem já experimentou demais, de quem já questionou demais para aceitar qualquer resposta como definitiva, qualquer verdade como absoluta, qualquer caminho como único, responde: ``Agora, precisamos aprender a viver sem as correntes, sem os ponteiros. A liberdade não está em destruir o tempo, mas em vivê-lo plenamente.'', palavras-mapa desenhando território novo, nomeação de espaço que surge apenas quando é nomeado, ``Precisamos aprender a sentir a vida como um rio que corre entre sertões e não como uma flecha disparada para o alvo predestinado. Livrar-se do tirano é apenas o primeiro passo; o verdadeiro desafio é não se tornar outro tirano no processo. Por isso a revolução nunca termina --- atravessa e é atravessada pelo próprio movimento que a origina.'' A boca fala porque as palavras estavam lá antes dele, sussurradas por algum antepassado que viu o tempo desdobrar-se em múltiplas dimensões.

O líder olha para ele, verdadeiramente olha, não apenas vê mas enxerga, não apenas registra mas compreende, não apenas observa mas reconhece, e em seu olhar há a luz do entendimento, da percepção súbita, da compreensão repentina que, como um raio em noite escura, ilumina brevemente a paisagem inteira, revelando conexões, relações, padrões invisíveis sob a luz normal, sob a percepção rotineira, sob a consciência habitual, ``Você fala como alguém que já viu muitas revoluções,'' diz ele, não como acusação, não como suspeita, não como confrontação, mas como reconhecimento de um igual, de um companheiro, de um aliado na interminável luta contra as simplificações, contra as reduções, contra as banalizações que são a verdadeira tirania, a verdadeira opressão, a verdadeira prisão da qual todos devemos tentar escapar.

Enquanto a poeira assenta e o caos começa a se organizar em uma nova ordem, não imposta de cima, não decretada por autoridade, não estabelecida por decreto, mas emergente do próprio processo de interação, de colaboração, de coexistência entre indivíduos livres, entre consciências autônomas, entre seres autodeterminados, o Relojoeiro percebe que sua jornada ainda não terminou, que o quebra-cabeça ainda não está completo, que a tapeçaria ainda não está tecida em sua totalidade, que o mapa ainda tem áreas em branco, regiões inexploradas, territórios desconhecidos que aguardam seu olhar, suas mãos, seu coração, sua mente para serem integrados à cartografia sempre expansiva, sempre inclusiva, sempre inacabada da compreensão. Ele agradece ao líder e aos revolucionários, deixando-os para reconstruir suas vidas em um novo ritmo, um ritmo que eles mesmos escolheram, um ritmo que não lhes é imposto por autoridade externa, por estrutura alienante, por sistema opressivo, mas que emerge de seus próprios corpos, de suas próprias necessidades, de seus próprios desejos, de suas próprias aspirações por uma existência mais autêntica, mais plena, mais verdadeira.

\bigskip

E assim, caro leitor, voltamos ao início, ao ponto onde o Relojoeiro, agora com as mãos sujas de poeira e o coração inflamado de novas convicções, reabre as cortinas de sua oficina --- \textit{``Mas o reabrir é um fingimento do abrir pela primeira vez? Ou o primeiro abrir já era repetição de outro mais antigo?''} --- pensamento que atravessa seus neurônios como relâmpago. Mas algo havia mudado. Ao lado da flor murcha da cidade invertida, do fragmento de rocha do vale, da folha do jardim e do cristal do Místico, agora repousa um pedaço do relógio destruído, um símbolo da revolução que não só desafiou o tempo, mas o reinventou, uma engrenagem torcida como uma pequena galáxia metálica, um fragmento de vidro que reflete não apenas a luz, mas todas as possibilidades que existem entre o ser e o não-ser, entre o acontecer e o não-acontecer, entre o lembrar e o esquecer, entre viver e deixar de existir.

Era uma manhã como outra qualquer, ou assim parecia ao Relojoeiro, que, ao abrir as cortinas de sua oficina, deixou que os primeiros raios de sol invadissem o espaço repleto de relógios e sombras. Mas algo dentro dele havia mudado, uma inquietação, um formigamento de pensamentos que não se aquietavam, que não se contentavam com o tic-tac monótono dos ponteiros. E foi então que ele decidiu, com a mesma certeza com que se sabe que o dia segue a noite, que era hora de buscar respostas, era hora de entender\ldots{}

E assim, caro leitor, continuamos nossa jornada, atravessando de margem a margem, sabendo que cada volta ao início é uma nova camada de compreensão, uma nova peça no quebra-cabeça infinito que é o tempo, cada retorno atravessa e é atravessado, espirais que refletem facetas do mesmo diamante primordial. Tempos são rios diferentes que atravessam o mesmo objeto atravessante. \textit{``O real não está no início nem no fim, ele se dispõe para a gente é no meio da travessia.''} Prepare-se, pois o próximo capítulo promete mais mistérios, mais revelações e, claro, mais voltas ao começo. Afinal, o tempo, essa entidade caprichosa, nonada pulsante, adora um bom enigma, tesouras do destino recortando figuras no papel amarelado dos dias, e nossos nomes se apagando lentamente dos registros, como neve que derrete ao sol da memória coletiva.

\begin{center}
\textit{A engrenagem quebrada da Revolução dos Ponteiros}
\end{center}
