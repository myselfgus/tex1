\chapter{A Cidade}

O Relojoeiro, agora um peregrino do tempo, vagava por caminhos que só um louco ou um poeta ousaria trilhar, e quem, afinal, consegue distinguir entre os dois quando a razão já se desfez como açúcar em água quente?, ele caminhava por estradas que não constavam em mapa algum, que talvez nem mesmo existissem antes de seus pés tocarem o solo, criando-as no próprio ato de percorrê-las, assim como o tempo talvez só exista quando o nomeamos, quando o dissecamos em horas, minutos, segundos, essa taxonomia ridícula que pretende domesticar o indomesticável, classificar o inclassificável, dar nome ao que transcende a própria linguagem, e não seria isso, pensava ele com um sorriso de ironia que seus lábios finos mal conseguiam conter, a maior das presunções humanas? acreditar que podemos aprisionar em caixinhas mecânicas aquilo que nos contém, aquilo que nos engole, aquilo que, no fim, nos aniquila com a mesma indiferença com que o mar apaga pegadas na areia.

A carta misteriosa, que agora carregava no bolso interno do casaco como um amuleto ou uma maldição, havia conduzido seus passos até aqui, até esta encruzilhada onde o vento parecia soprar de todas as direções simultaneamente e onde as sombras se comportavam de maneira estranha, desobedecendo às leis mais elementares da física, como se o sol, esse grande e impiedoso cronometrista, tivesse perdido a autoridade sobre aquilo que iluminava, e foi então que viu a criança, parada no meio da estrada como uma aparição, uma anomalia temporal materializada em forma humana, havia algo de profundamente perturbador naquela figura pequena, não apenas os cabelos brancos como neve, que poderiam ser explicados por alguma condição médica rara (o Relojoeiro gostava de explicações, de causas e efeitos, de respostas que se encaixassem perfeitamente nas perguntas, assim como as engrenagens se encaixam umas nas outras, precisas, previsíveis, controláveis), mas sobretudo os olhos, aqueles olhos que pareciam ter testemunhado o nascimento e a morte de impérios, olhos que carregavam o peso de milênios comprimidos em íris de um azul tão saturado e profundo que mais pareciam fragmentos de céu aprisionados em carne mortal.

``Bom dia, senhor que vem do lugar onde o tempo avança'', disse a criança, e sua voz era como o ruído de folhas secas arrastadas pelo vento, como páginas antigas sendo viradas em uma biblioteca silenciosa, uma voz que não pertencia àquele corpo pequeno, uma voz que havia viajado através de eras para encontrá-lo naquela encruzilhada sem nome, no limite entre o conhecido e o incognoscível, entre o que pode ser medido e o que só pode ser experienciado, o Relojoeiro, em outras circunstâncias, teria ficado perturbado com essa saudação enigmática, com essa presunção de conhecimento sobre sua origem e destino, mas a verdade é que após dias vagando por paisagens cada vez mais estranhas, onde a solidez das coisas parecia dissolver-se gradualmente, onde as leis da natureza pareciam perder sua autoridade à medida que ele se afastava da cidade dos relógios, seu ceticismo havia começado a ceder lugar a uma curiosidade quase infantil, aquela mesma curiosidade que o havia levado, décadas atrás, a desmontar o primeiro relógio só para ver como funcionava por dentro, só para entender o que fazia os ponteiros se moverem com aquela precisão hipnótica, com aquela certeza inabalável de quem nunca questiona seu propósito, sua existência, sua razão de ser.

``Você está procurando a Cidade das Horas Invertidas, não está?'', perguntou a criança, não como uma questão real, mas como uma confirmação do que já sabia, e o Relojoeiro, surpreendendo a si mesmo, respondeu que sim, embora até aquele exato momento não soubesse que existia tal lugar, muito menos que o estava procurando, mas havia algo na maneira como a criança falou o nome daquele lugar, algo na cadência das sílabas, na ressonância das palavras, que despertou nele uma certeza inexplicável, como se um órgão interno desconhecido, um sentido adormecido, de repente tivesse despertado e agora reconhecesse a verdade quando a ouvia, porque há mentiras que soam falsas mesmo quando revestidas de todos os argumentos lógicos, assim como há verdades que soam incontestáveis mesmo quando desafiam toda a racionalidade, e a existência da Cidade das Horas Invertidas claramente pertencia a esta segunda categoria, uma verdade tão elementar que ele se perguntava como havia passado tantos anos sem cogitar sua possibilidade, seu nome, sua existência.

``Então siga-me'', disse a criança, estendendo uma mão pequena e enrugada, uma mão que parecia simultaneamente jovem e antiga, macia como a de um bebê e calejada como a de um trabalhador após décadas de labor, e o Relojoeiro, num gesto que contradisse toda sua cautela habitual, toda sua desconfiança metódica contra o que não podia explicar (aquela mesma desconfiança que o havia transformado em excelente relojoeiro, capaz de diagnosticar os problemas mais obscuros, de resolver os enigmas mecânicos mais complexos), tomou aquela mão em sua própria e permitiu-se ser guiado por um ser que desafiava tudo o que ele considerava possível, que zombava silenciosamente de suas concepções sobre o mundo, sobre o funcionamento das coisas, sobre a própria natureza da realidade, e talvez por isso mesmo fosse o guia perfeito para o que estava por vir, porque há momentos em que só podemos avançar quando abandonamos tudo o que pensávamos saber, quando nos entregamos à vertigem do desconhecido com a mesma confiança cega de quem se lança em um abismo sabendo que aprenderá a voar durante a queda, ou que descobrirá, no último instante, que o abismo nunca existiu.

À medida que se aproximavam da cidade, o Relojoeiro começou a notar pequenas anomalias na paisagem, sutis a princípio, depois cada vez mais evidentes, inquietantes, como se a própria estrutura da realidade começasse a se desmanchar nas bordas, a revelar sua natureza ilusória, árvores cujas sombras pareciam mover-se na direção errada, pássaros que voavam em trajetórias impossíveis, como se estivessem retrocedendo no tempo mas avançando no espaço, riachos que corriam da foz para a nascente, um lapso de lógica tão absurdo que ele quase riu, teria rido se não houvesse algo de profundamente perturbador naquelas inversões, algo que sugeria não um erro da natureza, mas uma natureza inteiramente diferente, operando sob princípios que seu cérebro, formatado por décadas de causalidade linear, de tempo unidirecional, mal conseguia processar sem sentir aquela vertigem particular que acompanha os momentos em que nossas certezas mais fundamentais começam a desmoronar, em que o edifício inteiro do que chamamos conhecimento revela-se como a construção frágil e arbitrária que sempre foi.

``Bem-vindo à Cidade das Horas Invertidas'', disse a criança quando chegaram a um portal de pedra coberto por símbolos que pareciam tanto runas antigas quanto equações matemáticas avançadas, e talvez fossem ambas, pensou o Relojoeiro, talvez no início e no fim de todas as coisas, a magia primitiva e a ciência mais refinada sejam apenas manifestações diferentes do mesmo impulso humano de compreender, de nomear, de controlar o incontrolável, talvez os antigos místicos e os físicos quânticos estejam falando do mesmo mistério fundamental usando vocabulários diferentes, rituais diferentes, mas perseguindo a mesma presa esquiva, que sempre escapa por entre os dedos no exato momento em que pensamos tê-la capturado, e há algo de profundamente cômico nessa busca eterna, nessa perseguição inútil, como um cachorro correndo atrás do próprio rabo, girando em círculos cada vez mais rápidos sem jamais perceber a futilidade de seu esforço, a impossibilidade de seu objetivo.

A cidade que se revelava diante dele parecia, à primeira vista, uma cidade comum, com casas, ruas, praças, pessoas circulando em seus afazeres diários, mas logo o Relojoeiro percebeu que algo estava fundamentalmente errado, ou talvez não errado, mas diferente, operando segundo uma lógica alternativa que fazia com que o familiar parecesse estranho, e o estranho, familiar, porque os idosos caminhavam com a energia e a agilidade de crianças, enquanto as crianças se moviam com a lentidão cautelosa e a sabedoria silenciosa dos anciãos, os edifícios mais novos pareciam desgastados, enquanto as ruínas antigas brilhavam com a solidez e o frescor de construções recentes, os jardins exibiam flores que murchavam ao serem tocadas pela luz do sol, como se a vida fosse um fardo do qual finalmente se libertavam, enquanto nas sombras, protegidas da claridade, novas flores brotavam com uma exuberância que parecia quase obscena, uma explosão de cores e formas que celebrava não o nascimento, mas a morte próxima, não o crescimento, mas o inevitável declínio que aguarda tudo que existe, que respira, que pulsa.

A senhora que o recebeu na primeira casa, para onde a criança o conduziu antes de desaparecer com a mesma inexplicabilidade com que havia surgido, era uma mulher de aparência jovem, quase adolescente, mas que se movia com a dificuldade de alguém que carrega o peso de muitas décadas, apoiando-se em uma bengala entalhada com símbolos semelhantes aos do portal, seus olhos, como os da criança, tinham aquela qualidade desconcertante de antiguidade, de conhecimento acumulado através de experiências que transcendiam uma única vida humana. ``Bom dia, jovem senhor,'' disse ela, com uma voz que oscilava entre um soprano e um contralto, como se não conseguisse decidir em qual registro estabelecer-se, ``ou será boa noite? Nunca sei ao certo, pois aqui, como deve ter notado, as coisas não seguem exatamente a ordem à qual você está acostumado, aqui o tempo nos devolve o que perdemos, aqui recordamos o futuro e antecipamos o passado, aqui as feridas cicatrizam antes de serem abertas e as lágrimas secam antes de serem derramadas'', e havia nisso, pensou o Relojoeiro, uma espécie de crueldade disfarçada de bênção, porque que tipo de vida é essa onde já se conhece o desfecho de cada história antes mesmo de seu início? que tipo de existência é essa onde a surpresa, o mistério, a descoberta são substituídos por uma certeza que esvazia de significado cada ação, cada escolha, cada momento?

``Sente-se,'' continuou a senhora, indicando uma cadeira que parecia nova e antiga ao mesmo tempo, como se estivesse simultaneamente no início e no fim de sua existência como objeto, ``tome um pouco de chá'', e ela serviu de uma chaleira fumegante um líquido que, ao cair na xícara, mudava de cor do escuro para o claro, do opaco para o transparente, ``e me conte o que o traz à nossa cidade, embora, é claro, eu já saiba a resposta, assim como sei o que dirá antes mesmo que você formule as palavras em sua mente'', e o Relojoeiro, irritado com essa presunção, com essa arrogância disfarçada de onisciência, decidiu mentir, dizer algo completamente inesperado, algo que nem ele mesmo havia pensado até aquele exato momento, mas quando abriu a boca, as palavras que saíram foram exatamente aquelas que a senhora já esperava, ``Vim em busca de respostas sobre o tempo'', e havia algo de humilhante em ser tão previsível, tão transparente, em não conseguir surpreender nem a si mesmo, porque talvez seja essa a maior armadilha do tempo linear ao qual estamos habituados, a ilusão de que somos livres, de que nossas escolhas são realmente nossas, quando na verdade talvez sejamos apenas atores seguindo um roteiro escrito muito antes de nascermos, marionetes dançando na ponta de fios invisíveis manipulados por mãos que não podemos ver, que nem mesmo sabemos se existem.

\begin{center}
\textit{``O tempo não é o que parece''}
\end{center}

A senhora sorriu, um sorriso que continha em si todas as ironias do universo, todos os paradoxos do tempo, todas as contradições da existência, ``Ah, a carta. Sim, conheço bem essa mensagem. Muitos vieram antes de você, seguindo essas mesmas palavras, e muitos virão depois, porque essa é a natureza do tempo invertido: o que parece único é sempre repetição, e o que parece repetição é sempre único'', e enquanto falava, seus olhos pareciam ver não apenas o homem à sua frente, mas todas as versões dele que existiram e existirão, todas as possibilidades que se ramificam a partir de cada escolha, cada pensamento, cada respiração, como se o tempo, neste lugar, não fosse uma linha nem mesmo um círculo, mas uma esfera infinitamente complexa onde todos os pontos se conectam a todos os outros, onde qualquer momento pode ser acessado a partir de qualquer outro momento, onde a própria noção de antes e depois perde todo o sentido.

``E o que descobriram, esses outros que vieram antes?'', perguntou o Relojoeiro, genuinamente curioso agora, sua irritação inicial substituída por uma fome de conhecimento que não sentia desde os dias de juventude, quando cada relógio desmontado era um universo a ser explorado, cada engrenagem um mistério a ser desvendado.

``Descobriram o que você descobrirá'', respondeu a senhora, e havia em sua voz uma nota de compaixão que o Relojoeiro não esperava, uma suavidade que contrastava com a dureza de suas palavras anteriores, ``que o tempo não pode ser compreendido, apenas experienciado. Que todas as teorias, todas as filosofias, todas as ciências são apenas mapas aproximados de um território infinito. Que a verdade sobre o tempo é como a água: assume a forma do recipiente que a contém, mas sua essência permanece sempre a mesma, sempre misteriosa, sempre além do alcance de nossas mãos'', e com essas palavras, ela estendeu uma flor murcha, uma flor que parecia ter morrido há séculos mas que ainda mantinha um resquício de cor, um eco de perfume, uma memória de beleza, ``leve isto como lembrança de sua passagem por nossa cidade. Quando olhar para ela, lembre-se de que a morte não é o oposto da vida, mas parte dela. Que o fim está contido no início, e o início, no fim. Que o tempo invertido não é a negação do tempo, mas sua revelação mais profunda''.

O Relojoeiro aceitou a flor, sentindo em suas pétalas secas o peso de verdades que ainda não compreendia completamente, de revelações que levariam anos, talvez décadas, talvez toda uma vida para serem assimiladas, e quando finalmente deixou a cidade, caminhando de volta pelo portal de símbolos, pelo caminho de anomalias, pela estrada que talvez não existisse fora de sua própria necessidade de percorrê-la, carregava consigo não apenas uma flor murcha, mas uma nova perspectiva, uma nova maneira de olhar para o tempo que tanto tentara dominar, uma humildade recém-descoberta diante do mistério que dedicara sua existência a resolver sem jamais suspeitara que a solução pudesse estar não na resposta, mas na própria pergunta, não na compreensão, mas na aceitação, não no controle, mas na rendição.

\bigskip

E assim, caro leitor, voltamos ao início, ao ponto onde o Relojoeiro, agora com uma flor murcha no bolso e uma nova ferida na alma, reabre as cortinas de sua oficina. Mas algo mudou. O tic-tac dos relógios parece diferente agora, menos uma prisão e mais uma música, menos uma sentença e mais um convite. Ele olha para os ponteiros que giram, giram, giram, e pela primeira vez se pergunta: para onde vão eles, realmente? E de onde vêm? E será que importa, afinal, quando o destino e a origem são apenas nomes diferentes para o mesmo mistério?

\begin{center}
\textit{A flor murcha da Cidade das Horas Invertidas}
\end{center}
