%%%%%%%%%%%%%%%%%%%% O Relojoeiro - Formato Springer %%%%%%%%%%%%%%%%%%%%%%%%%%%%%
%
% Baseado no template book.tex (Springer Nature)
%
%%%%%%%%%%%%%%%% Springer Nature %%%%%%%%%%%%%%%%%%%%%%%%%%


% RECOMMENDED %%%%%%%%%%%%%%%%%%%%%%%%%%%%%%%%%%%%%%%%%%%%%%%%%%%
\documentclass[graybox,envcountchap,sectrefs]{svmono}

\usepackage[utf8]{inputenc}
\usepackage[T1]{fontenc}

\usepackage{type1cm}

\usepackage{makeidx}         % allows index generation
\usepackage{graphicx}        % standard LaTeX graphics tool
                             % when including figure files
\usepackage{multicol}        % used for the two-column index
\usepackage[bottom]{footmisc}% places footnotes at page bottom

\usepackage{newtxtext}       %
\usepackage[varvw]{newtxmath}       % selects Times Roman as basic font

\makeindex             % used for the subject index
                       % please use the style svind.ist with
                       % your makeindex program

%%%%%%%%%%%%%%%%%%%%%%%%%%%%%%%%%%%%%%%%%%%%%%%%%%%%%%%%%%%%%%%%%%%%%

\begin{document}

\author{Gustavo Mendes e Silva}
\title{O Relojoeiro}
\subtitle{Uma Jornada Através do Tempo}
\maketitle

\frontmatter%%%%%%%%%%%%%%%%%%%%%%%%%%%%%%%%%%%%%%%%%%%%%%%%%%%%%%

%% Dedicatória
\chapter*{Dedicatória}
\begin{center}
\textit{Aos que buscam o tempo,\\e aos que pelo tempo são buscados.}
\end{center}

%% Prefácio
\chapter*{Prefácio}

Esta obra nasceu de uma inquietação fundamental: o que é o tempo? Não o tempo dos relógios, não o tempo dos calendários, não o tempo das agendas e compromissos, mas o tempo como substância primordial da existência, como tecido no qual bordamos nossas vidas, como oceano no qual flutuamos desde o nascimento até a morte.

O Relojoeiro é uma alegoria, uma meditação ficcional sobre nossa relação com a temporalidade. Através de sua jornada por paisagens impossíveis --- cidades onde o tempo flui ao contrário, vales onde nada muda, jardins onde memórias são cultivadas como plantas, teatros onde filósofos debatem eternamente, praças onde revoluções explodem contra a tirania dos ponteiros --- exploramos as múltiplas faces daquilo que chamamos, por falta de nome melhor, de tempo.

Cada capítulo representa uma perspectiva diferente, uma tentativa distinta de apreender o inapreensível. Juntos, formam um mosaico incompleto, um quebra-cabeça cujas peças não se encaixam perfeitamente, porque o tempo, em sua essência, resiste a qualquer tentativa de totalização, de sistematização, de domesticação.

\vspace{1cm}
\begin{flushright}
\textit{O Autor}
\end{flushright}

%% Epígrafe
\chapter*{Epígrafe}
\begin{center}
\Large\textit{``O tempo não é o que parece.''}
\end{center}

\tableofcontents

\mainmatter%%%%%%%%%%%%%%%%%%%%%%%%%%%%%%%%%%%%%%%%%%%%%%%%%%%%%%%

%%%%%%%%%%%%%%%%%%%%%%%%%%%%%%%%%%%%%%%%%%%%%%%%%%%%%%%%%%%%%%%%%%%%%
% PARTE I - A PARTIDA
%%%%%%%%%%%%%%%%%%%%%%%%%%%%%%%%%%%%%%%%%%%%%%%%%%%%%%%%%%%%%%%%%%%%%

\part{A Partida}

\input{capitulos/cap1-relojoeiro}

\chapter{A Cidade}

O Relojoeiro, agora um peregrino do tempo, vagava por caminhos que só um louco ou um poeta ousaria trilhar, e quem, afinal, consegue distinguir entre os dois quando a razão já se desfez como açúcar em água quente?, ele caminhava por estradas que não constavam em mapa algum, que talvez nem mesmo existissem antes de seus pés tocarem o solo, criando-as no próprio ato de percorrê-las, assim como o tempo talvez só exista quando o nomeamos, quando o dissecamos em horas, minutos, segundos, essa taxonomia ridícula que pretende domesticar o indomesticável, classificar o inclassificável, dar nome ao que transcende a própria linguagem, e não seria isso, pensava ele com um sorriso de ironia que seus lábios finos mal conseguiam conter, a maior das presunções humanas? acreditar que podemos aprisionar em caixinhas mecânicas aquilo que nos contém, aquilo que nos engole, aquilo que, no fim, nos aniquila com a mesma indiferença com que o mar apaga pegadas na areia.

A carta misteriosa, que agora carregava no bolso interno do casaco como um amuleto ou uma maldição, havia conduzido seus passos até aqui, até esta encruzilhada onde o vento parecia soprar de todas as direções simultaneamente e onde as sombras se comportavam de maneira estranha, desobedecendo às leis mais elementares da física, como se o sol, esse grande e impiedoso cronometrista, tivesse perdido a autoridade sobre aquilo que iluminava, e foi então que viu a criança, parada no meio da estrada como uma aparição, uma anomalia temporal materializada em forma humana, havia algo de profundamente perturbador naquela figura pequena, não apenas os cabelos brancos como neve, que poderiam ser explicados por alguma condição médica rara (o Relojoeiro gostava de explicações, de causas e efeitos, de respostas que se encaixassem perfeitamente nas perguntas, assim como as engrenagens se encaixam umas nas outras, precisas, previsíveis, controláveis), mas sobretudo os olhos, aqueles olhos que pareciam ter testemunhado o nascimento e a morte de impérios, olhos que carregavam o peso de milênios comprimidos em íris de um azul tão saturado e profundo que mais pareciam fragmentos de céu aprisionados em carne mortal.

``Bom dia, senhor que vem do lugar onde o tempo avança'', disse a criança, e sua voz era como o ruído de folhas secas arrastadas pelo vento, como páginas antigas sendo viradas em uma biblioteca silenciosa, uma voz que não pertencia àquele corpo pequeno, uma voz que havia viajado através de eras para encontrá-lo naquela encruzilhada sem nome, no limite entre o conhecido e o incognoscível, entre o que pode ser medido e o que só pode ser experienciado, o Relojoeiro, em outras circunstâncias, teria ficado perturbado com essa saudação enigmática, com essa presunção de conhecimento sobre sua origem e destino, mas a verdade é que após dias vagando por paisagens cada vez mais estranhas, onde a solidez das coisas parecia dissolver-se gradualmente, onde as leis da natureza pareciam perder sua autoridade à medida que ele se afastava da cidade dos relógios, seu ceticismo havia começado a ceder lugar a uma curiosidade quase infantil, aquela mesma curiosidade que o havia levado, décadas atrás, a desmontar o primeiro relógio só para ver como funcionava por dentro, só para entender o que fazia os ponteiros se moverem com aquela precisão hipnótica, com aquela certeza inabalável de quem nunca questiona seu propósito, sua existência, sua razão de ser.

``Você está procurando a Cidade das Horas Invertidas, não está?'', perguntou a criança, não como uma questão real, mas como uma confirmação do que já sabia, e o Relojoeiro, surpreendendo a si mesmo, respondeu que sim, embora até aquele exato momento não soubesse que existia tal lugar, muito menos que o estava procurando, mas havia algo na maneira como a criança falou o nome daquele lugar, algo na cadência das sílabas, na ressonância das palavras, que despertou nele uma certeza inexplicável, como se um órgão interno desconhecido, um sentido adormecido, de repente tivesse despertado e agora reconhecesse a verdade quando a ouvia, porque há mentiras que soam falsas mesmo quando revestidas de todos os argumentos lógicos, assim como há verdades que soam incontestáveis mesmo quando desafiam toda a racionalidade, e a existência da Cidade das Horas Invertidas claramente pertencia a esta segunda categoria, uma verdade tão elementar que ele se perguntava como havia passado tantos anos sem cogitar sua possibilidade, seu nome, sua existência.

``Então siga-me'', disse a criança, estendendo uma mão pequena e enrugada, uma mão que parecia simultaneamente jovem e antiga, macia como a de um bebê e calejada como a de um trabalhador após décadas de labor, e o Relojoeiro, num gesto que contradisse toda sua cautela habitual, toda sua desconfiança metódica contra o que não podia explicar (aquela mesma desconfiança que o havia transformado em excelente relojoeiro, capaz de diagnosticar os problemas mais obscuros, de resolver os enigmas mecânicos mais complexos), tomou aquela mão em sua própria e permitiu-se ser guiado por um ser que desafiava tudo o que ele considerava possível, que zombava silenciosamente de suas concepções sobre o mundo, sobre o funcionamento das coisas, sobre a própria natureza da realidade, e talvez por isso mesmo fosse o guia perfeito para o que estava por vir, porque há momentos em que só podemos avançar quando abandonamos tudo o que pensávamos saber, quando nos entregamos à vertigem do desconhecido com a mesma confiança cega de quem se lança em um abismo sabendo que aprenderá a voar durante a queda, ou que descobrirá, no último instante, que o abismo nunca existiu.

À medida que se aproximavam da cidade, o Relojoeiro começou a notar pequenas anomalias na paisagem, sutis a princípio, depois cada vez mais evidentes, inquietantes, como se a própria estrutura da realidade começasse a se desmanchar nas bordas, a revelar sua natureza ilusória, árvores cujas sombras pareciam mover-se na direção errada, pássaros que voavam em trajetórias impossíveis, como se estivessem retrocedendo no tempo mas avançando no espaço, riachos que corriam da foz para a nascente, um lapso de lógica tão absurdo que ele quase riu, teria rido se não houvesse algo de profundamente perturbador naquelas inversões, algo que sugeria não um erro da natureza, mas uma natureza inteiramente diferente, operando sob princípios que seu cérebro, formatado por décadas de causalidade linear, de tempo unidirecional, mal conseguia processar sem sentir aquela vertigem particular que acompanha os momentos em que nossas certezas mais fundamentais começam a desmoronar, em que o edifício inteiro do que chamamos conhecimento revela-se como a construção frágil e arbitrária que sempre foi.

``Bem-vindo à Cidade das Horas Invertidas'', disse a criança quando chegaram a um portal de pedra coberto por símbolos que pareciam tanto runas antigas quanto equações matemáticas avançadas, e talvez fossem ambas, pensou o Relojoeiro, talvez no início e no fim de todas as coisas, a magia primitiva e a ciência mais refinada sejam apenas manifestações diferentes do mesmo impulso humano de compreender, de nomear, de controlar o incontrolável, talvez os antigos místicos e os físicos quânticos estejam falando do mesmo mistério fundamental usando vocabulários diferentes, rituais diferentes, mas perseguindo a mesma presa esquiva, que sempre escapa por entre os dedos no exato momento em que pensamos tê-la capturado, e há algo de profundamente cômico nessa busca eterna, nessa perseguição inútil, como um cachorro correndo atrás do próprio rabo, girando em círculos cada vez mais rápidos sem jamais perceber a futilidade de seu esforço, a impossibilidade de seu objetivo.

A cidade que se revelava diante dele parecia, à primeira vista, uma cidade comum, com casas, ruas, praças, pessoas circulando em seus afazeres diários, mas logo o Relojoeiro percebeu que algo estava fundamentalmente errado, ou talvez não errado, mas diferente, operando segundo uma lógica alternativa que fazia com que o familiar parecesse estranho, e o estranho, familiar, porque os idosos caminhavam com a energia e a agilidade de crianças, enquanto as crianças se moviam com a lentidão cautelosa e a sabedoria silenciosa dos anciãos, os edifícios mais novos pareciam desgastados, enquanto as ruínas antigas brilhavam com a solidez e o frescor de construções recentes, os jardins exibiam flores que murchavam ao serem tocadas pela luz do sol, como se a vida fosse um fardo do qual finalmente se libertavam, enquanto nas sombras, protegidas da claridade, novas flores brotavam com uma exuberância que parecia quase obscena, uma explosão de cores e formas que celebrava não o nascimento, mas a morte próxima, não o crescimento, mas o inevitável declínio que aguarda tudo que existe, que respira, que pulsa.

A senhora que o recebeu na primeira casa, para onde a criança o conduziu antes de desaparecer com a mesma inexplicabilidade com que havia surgido, era uma mulher de aparência jovem, quase adolescente, mas que se movia com a dificuldade de alguém que carrega o peso de muitas décadas, apoiando-se em uma bengala entalhada com símbolos semelhantes aos do portal, seus olhos, como os da criança, tinham aquela qualidade desconcertante de antiguidade, de conhecimento acumulado através de experiências que transcendiam uma única vida humana. ``Bom dia, jovem senhor,'' disse ela, com uma voz que oscilava entre um soprano e um contralto, como se não conseguisse decidir em qual registro estabelecer-se, ``ou será boa noite? Nunca sei ao certo, pois aqui, como deve ter notado, as coisas não seguem exatamente a ordem à qual você está acostumado, aqui o tempo nos devolve o que perdemos, aqui recordamos o futuro e antecipamos o passado, aqui as feridas cicatrizam antes de serem abertas e as lágrimas secam antes de serem derramadas'', e havia nisso, pensou o Relojoeiro, uma espécie de crueldade disfarçada de bênção, porque que tipo de vida é essa onde já se conhece o desfecho de cada história antes mesmo de seu início? que tipo de existência é essa onde a surpresa, o mistério, a descoberta são substituídos por uma certeza que esvazia de significado cada ação, cada escolha, cada momento?

``Sente-se,'' continuou a senhora, indicando uma cadeira que parecia nova e antiga ao mesmo tempo, como se estivesse simultaneamente no início e no fim de sua existência como objeto, ``tome um pouco de chá'', e ela serviu de uma chaleira fumegante um líquido que, ao cair na xícara, mudava de cor do escuro para o claro, do opaco para o transparente, ``e me conte o que o traz à nossa cidade, embora, é claro, eu já saiba a resposta, assim como sei o que dirá antes mesmo que você formule as palavras em sua mente'', e o Relojoeiro, irritado com essa presunção, com essa arrogância disfarçada de onisciência, decidiu mentir, dizer algo completamente inesperado, algo que nem ele mesmo havia pensado até aquele exato momento, mas quando abriu a boca, as palavras que saíram foram exatamente aquelas que a senhora já esperava, ``Vim em busca de respostas sobre o tempo'', e havia algo de humilhante em ser tão previsível, tão transparente, em não conseguir surpreender nem a si mesmo, porque talvez seja essa a maior armadilha do tempo linear ao qual estamos habituados, a ilusão de que somos livres, de que nossas escolhas são realmente nossas, quando na verdade talvez sejamos apenas atores seguindo um roteiro escrito muito antes de nascermos, marionetes dançando na ponta de fios invisíveis manipulados por mãos que não podemos ver, que nem mesmo sabemos se existem.

\begin{center}
\textit{``O tempo não é o que parece''}
\end{center}

A senhora sorriu, um sorriso que continha em si todas as ironias do universo, todos os paradoxos do tempo, todas as contradições da existência, ``Ah, a carta. Sim, conheço bem essa mensagem. Muitos vieram antes de você, seguindo essas mesmas palavras, e muitos virão depois, porque essa é a natureza do tempo invertido: o que parece único é sempre repetição, e o que parece repetição é sempre único'', e enquanto falava, seus olhos pareciam ver não apenas o homem à sua frente, mas todas as versões dele que existiram e existirão, todas as possibilidades que se ramificam a partir de cada escolha, cada pensamento, cada respiração, como se o tempo, neste lugar, não fosse uma linha nem mesmo um círculo, mas uma esfera infinitamente complexa onde todos os pontos se conectam a todos os outros, onde qualquer momento pode ser acessado a partir de qualquer outro momento, onde a própria noção de antes e depois perde todo o sentido.

``E o que descobriram, esses outros que vieram antes?'', perguntou o Relojoeiro, genuinamente curioso agora, sua irritação inicial substituída por uma fome de conhecimento que não sentia desde os dias de juventude, quando cada relógio desmontado era um universo a ser explorado, cada engrenagem um mistério a ser desvendado.

``Descobriram o que você descobrirá'', respondeu a senhora, e havia em sua voz uma nota de compaixão que o Relojoeiro não esperava, uma suavidade que contrastava com a dureza de suas palavras anteriores, ``que o tempo não pode ser compreendido, apenas experienciado. Que todas as teorias, todas as filosofias, todas as ciências são apenas mapas aproximados de um território infinito. Que a verdade sobre o tempo é como a água: assume a forma do recipiente que a contém, mas sua essência permanece sempre a mesma, sempre misteriosa, sempre além do alcance de nossas mãos'', e com essas palavras, ela estendeu uma flor murcha, uma flor que parecia ter morrido há séculos mas que ainda mantinha um resquício de cor, um eco de perfume, uma memória de beleza, ``leve isto como lembrança de sua passagem por nossa cidade. Quando olhar para ela, lembre-se de que a morte não é o oposto da vida, mas parte dela. Que o fim está contido no início, e o início, no fim. Que o tempo invertido não é a negação do tempo, mas sua revelação mais profunda''.

O Relojoeiro aceitou a flor, sentindo em suas pétalas secas o peso de verdades que ainda não compreendia completamente, de revelações que levariam anos, talvez décadas, talvez toda uma vida para serem assimiladas, e quando finalmente deixou a cidade, caminhando de volta pelo portal de símbolos, pelo caminho de anomalias, pela estrada que talvez não existisse fora de sua própria necessidade de percorrê-la, carregava consigo não apenas uma flor murcha, mas uma nova perspectiva, uma nova maneira de olhar para o tempo que tanto tentara dominar, uma humildade recém-descoberta diante do mistério que dedicara sua existência a resolver sem jamais suspeitara que a solução pudesse estar não na resposta, mas na própria pergunta, não na compreensão, mas na aceitação, não no controle, mas na rendição.

\bigskip

E assim, caro leitor, voltamos ao início, ao ponto onde o Relojoeiro, agora com uma flor murcha no bolso e uma nova ferida na alma, reabre as cortinas de sua oficina. Mas algo mudou. O tic-tac dos relógios parece diferente agora, menos uma prisão e mais uma música, menos uma sentença e mais um convite. Ele olha para os ponteiros que giram, giram, giram, e pela primeira vez se pergunta: para onde vão eles, realmente? E de onde vêm? E será que importa, afinal, quando o destino e a origem são apenas nomes diferentes para o mesmo mistério?

\begin{center}
\textit{A flor murcha da Cidade das Horas Invertidas}
\end{center}


%%%%%%%%%%%%%%%%%%%%%%%%%%%%%%%%%%%%%%%%%%%%%%%%%%%%%%%%%%%%%%%%%%%%%
% PARTE II - A JORNADA
%%%%%%%%%%%%%%%%%%%%%%%%%%%%%%%%%%%%%%%%%%%%%%%%%%%%%%%%%%%%%%%%%%%%%

\part{A Jornada}

\chapter{O Vale}

Voltando ao início mais uma vez, nosso Relojoeiro reabre as cortinas da oficina com um olhar que agora mistura a curiosidade de um explorador e o cansaço de um velho marinheiro, ou talvez não seja realmente cansaço o que sente, mas aquela peculiar forma de lucidez que chega quando todas as ilusões começam a desmoronar, quando os alicerces que sustentavam sua compreensão do mundo revelam-se tão frágeis quanto casas de papel em uma tempestade, é mais um despojamento, uma nudez essencial que o deixa vulnerável às verdades que sempre estiveram ali, escondidas sob o véu da cotidianidade, da rotina, da certeza não examinada, e é com essa nova vulnerabilidade, esse novo estado de receptividade quase dolorosa, que ele decide seguir adiante, não para trás em direção à segurança ilusória das antigas crenças, não para os lados em tentativas desesperadas de evitar o inevitável, mas para um destino que promete, se não respostas, ao menos perguntas mais precisas, mais essenciais, e não é isso, afinal, o início de toda verdadeira sabedoria? não questionar mais, mas questionar melhor?

Fechando a oficina, ele guarda a chave no mesmo bolso onde mantém a carta misteriosa e a flor murcha da Cidade das Horas Invertidas, objetos que agora funcionam como talismãs ou âncoras em um mundo que se torna cada vez mais fluido, cada vez mais resistente às categorizações fáceis, aos nomes convencionais, às explicações consagradas que os seres humanos inventam não para compreender o universo, mas para protegerem-se da vertigem de sua incompreensibilidade fundamental, pequenas mentiras consensuais que permitem seguir adiante sem confrontar o abismo que existe sob cada passo, sob cada pensamento, sob cada respiração, o Relojoeiro sente o peso desses objetos contra seu corpo como um lembrete constante do que deixou para trás e do que ainda busca, são pontos de contato com o real que o impedem de flutuar completamente para fora de si mesmo, para longe de tudo o que um dia considerou verdadeiro, e há uma ironia perversa nisso, pensa ele com um sorriso que ninguém vê, que uma flor morta colhida em uma cidade impossível seja agora um de seus poucos vínculos com a realidade, uma prova tangível de que o que experimenta não é apenas um sonho elaborado, uma alucinação coerente, um delírio sistemático produzido por uma mente saturada de tempo, intoxicada por segundos, minutos, horas.

O caminho que se desenrola à sua frente não consta em nenhum mapa, não respeita nenhuma coordenada geográfica conhecida, parece existir apenas como uma função de sua própria necessidade de percorrê-lo, uma manifestação física de um impulso interno, como se a estrada fosse o prolongamento visível de seus questionamentos mais profundos, de suas dúvidas mais essenciais, e não deveria surpreendê-lo, pois se o tempo já não é o que parecia ser, por que o espaço manteria sua natureza convencional, sua obediência às leis da física? por que qualquer aspecto da realidade permaneceria intacto quando sua componente mais fundamental, aquela que permeia e sustenta todas as outras, é revelada como algo completamente diferente do que sempre supusemos? aquele que questiona o tempo deve estar preparado para que todas as outras certezas desmoronem em sequência, como dominós alinhados com precisão maníaca, pois nenhuma verdade existe isolada, nenhum conceito sobrevive à reformulação completa de seu contexto, e o contexto último de toda experiência humana não é o espaço, não é a matéria, não é a energia, mas o tempo, esse contínuo inescapável que nos atravessa, nos constitui, nos delimita.

À medida que avança, observa que a paisagem se torna cada vez mais estática, como se um glacialista invisível houvesse reduzido a temperatura do mundo até quase o zero absoluto, congelando o fluxo natural das coisas, suspendendo os processos mais básicos, árvores que parecem esculpidas em uma substância que imita a vida sem possuí-la realmente, folhas que não tremem mesmo quando o vento sopra (mas haverá realmente vento aqui?), riachos cujas águas são tão imóveis que mais parecem vidro polido, reflexos perfeitos que nunca se distorcem, que nunca vibram com aquela imprecisão característica de tudo que é vivo, e o Relojoeiro se pergunta se ainda está no mesmo mundo que conheceu, ou se atravessou sem perceber alguma fronteira invisível para um território onde as leis fundamentais da física, da biologia, do tempo operem segundo princípios completamente diferentes, ou não operem de forma alguma.

Quando finalmente avista o Vale da Imutabilidade, o nome surge espontaneamente em sua consciência como se tivesse sido sempre óbvio, como se não pudesse existir outro título para este lugar, este não-lugar, esta suspensão da lógica convencional em favor de uma outra lógica que ainda não compreende completamente mas pode intuir, e há nessa intuição um elemento de terror ancestral, como se alguma parte primitiva de sua mente reconhecesse um perigo mais fundamental do que qualquer ameaça física, um perigo para a própria coerência de sua identidade, de sua existência como ser no tempo, o vale se estende abaixo dele, uma depressão geográfica perfeitamente simétrica, como se tivesse sido esculpida por um artista obsessivo e não pelos processos caóticos da erosão, do movimento tectônico, das forças cegas que normalmente moldam as paisagens, e no fundo do vale, dispostas com uma ordem que sugere intenção, planejamento, vontade consciente, as habitações dos que escolheram (ou foram escolhidos para) viver na imutabilidade.

O primeiro encontro é tão repentino quanto inexplicável, pois não viu nem ouviu ninguém se aproximar, não houve passos, respiração, o ruído característico de tecidos em movimento, nenhum dos pequenos sinais que anunciam a presença humana, apenas a súbita materialização de um homem à sua frente, ou talvez não tenha sido súbita, talvez tenha sido gradual e o Relojoeiro, habituado ao ritmo acelerado do mundo convencional, simplesmente não tenha notado o lento surgimento dessa figura alta, ereta, impassível como uma estátua grega, talvez o homem sempre tenha estado ali e foi o Relojoeiro quem, por razões que nem mesmo ele compreende, apenas agora se tornou capaz de percebê-lo, de registrar sua presença, e há nessa possibilidade algo tão perturbador quanto na primeira hipótese, pois sugere não apenas um mundo diferente do que conhece, mas um mundo fundamentalmente inapreensível, onde nem mesmo seus sentidos, essas janelas primárias para a realidade, são confiáveis.

``Bem-vindo ao Vale da Imutabilidade,'' diz uma voz grave e profunda, que parece emanar não apenas do homem à sua frente, mas das próprias pedras, do próprio ar, como se todo o ambiente fosse uma extensão daquele ser, ou como se aquele ser fosse uma manifestação localizada, uma concentração do ambiente, ``aqui, o tempo é um conceito ultrapassado, um luxo que decidimos dispensar'', e o homem não se move enquanto fala, seu rosto permanece tão estático quanto as rochas, seus olhos fixos, quase vítreos, e no entanto transmitindo um tipo de inteligência, de consciência, que o Relojoeiro reconhece imediatamente como não-humana, não porque seja inferior à humanidade, mas porque a transcende, porque existe segundo padrões que nossas mentes, moldadas por milênios de evolução em um ambiente temporal específico, não foram equipadas para compreender, ``por que correr, por que mudar, quando se pode simplesmente existir?'', e há nessa pergunta uma provocação, um desafio à própria essência do que o Relojoeiro sempre considerou valioso, essencial, verdadeiro.

``Como vocês conseguem viver assim, presos em um eterno agora?'', pergunta ele, com seu habitual sarcasmo que agora funciona como uma frágil proteção contra o desconforto profundo que sente diante dessa negação radical de tudo o que define a experiência humana, pois se não há tempo, não há história, nem memória, nem expectativa, nem narrativa, nem aquela tensão fundamental entre o que fomos e o que seremos que constitui a própria textura da identidade pessoal, se não há tempo não há transformação, não há possibilidade, não há escolha, não há liberdade, não há propósito, não há sentido no sentido humano do termo, e como pode algo que não muda ser considerado vivo? não é a capacidade de adaptação, de transformação, de aprendizado, de crescimento, a própria definição do que distingue a vida da não-vida? não é o movimento no tempo a condição sine qua non da existência consciente?

O homem sorri, um sorriso que mais parece um esgar, um arranjo de músculos faciais que imita o que nos seres humanos seria uma expressão de satisfação, de alegria, de compreensão, mas que nele transmite apenas uma espécie de condescendência, de superioridade, de distanciamento irônico, ``a mudança é uma ilusão, uma necessidade fabricada por mentes inquietas, aqui, encontramos paz na constância, na permanência, a dor não nos toca, a alegria não nos move, apenas somos'', e enquanto fala, o homem permanece perfeitamente imóvel, nem mesmo seus lábios se movem, como se a voz fosse uma propriedade do ambiente e não do indivíduo, como se o ar vibrasse espontaneamente em padrões que o cérebro do Relojoeiro interpreta como palavras, como linguagem, como comunicação, embora em um nível mais profundo ele suspeite que o que ocorre não seja realmente comunicação, mas algo completamente diferente, algo para o qual não temos nome nem conceito.

O Relojoeiro observa os habitantes do vale, e percebe que todos compartilham aquela mesma qualidade inquietante, aquela mesma negação do movimento natural, do fluxo esperado, eles se movem, sim, realizam ações, interagem com objetos e entre si, mas seus movimentos parecem ensaiados, predeterminados, executados com a precisão mecânica de autômatos, de marionetes manipuladas por uma consciência central que poderia ser um deus, um programa de computador, um campo de força, um consenso coletivo, um algoritmo, ou nada disso, algo tão distante de nossas categorias que sequer podemos começar a conceituá-lo sem distorcê-lo completamente, há uma perfeição em cada gesto que o torna, paradoxalmente, menos real, como se a própria ausência de erros, de hesitações, de desvios, fosse a prova mais contundente de sua artificialidade, de sua não-humanidade, é como assistir a uma peça onde os atores estão tão comprometidos com seus papéis, tão imersos em suas performances, que perderam completamente o contato com a espontaneidade, com a imprevisibilidade, com a vulnerabilidade que caracteriza a vida fora do palco.

Entre os habitantes, no entanto, uma mulher se destaca, e o Relojoeiro a identifica imediatamente, como se já soubesse que a encontraria ali, como se sua presença fosse uma necessidade narrativa, um ponto de inflexão em uma história que ele não está apenas testemunhando, mas de alguma forma criando à medida que avança, seus olhos, ao contrário dos outros, brilham com uma luz de inquietação, como pequenas fissuras no tecido estático da imutabilidade, pequenas insurreições contra o regime de permanência que governa o vale, há nela uma humanidade que parece ter sido drenada dos demais, uma conexão preservada com o fluxo, com a incerteza, com a busca que caracteriza os seres que existem no tempo e não fora dele, ela se aproxima do Relojoeiro e, com uma voz que parece um sussurro carregado pelo vento, tão diferente da voz inumana do primeiro homem, diz: ``aqui, tudo é perfeito, mas eu sinto falta do imperfeito. A imutabilidade é uma prisão, e eu anseio por algo mais'', e nessas palavras simples, nessa confissão de insuficiência, de incompletude, há mais sinceridade, mais vulnerabilidade, mais humanidade do que em todos os discursos polidos, precisos, incontestáveis dos demais habitantes do vale.

``E por que não deixa este lugar?'', pergunta o Relojoeiro, sabendo que a resposta não será simples, que nada nesta jornada tem sido ou será simples, que a complexidade, a ambiguidade, a contradição são as únicas constantes em um universo onde todas as outras certezas são reveladas como provisórias, como contingentes, como acidentais, ele formulou a pergunta sabendo que ela é simultaneamente óbvia e impossível, pois quem, tendo conhecido a imutabilidade, poderia simplesmente afastar-se dela sem ser transformado, sem ser marcado de forma indelével por essa experiência? quem, tendo contemplado a eternidade, poderia retornar ao tempo sem carregar consigo para sempre a nostalgia do infinito, a memória da perfeição, a cicatriz da totalidade?

``A saída não é permitida'', responde ela, e em sua voz há uma resignação que o Relojoeiro reconhece como exclusivamente humana, esse reconhecimento da própria impotência diante de forças maiores, esse confronto com limitações que não podem ser superadas pela vontade, pela inteligência, pelo esforço, essa aceitação relutante do que não pode ser mudado que constitui, paradoxalmente, o início de toda possível liberdade, ``o Vale da Imutabilidade não tolera deserções. Quem tenta partir é condenado ao esquecimento'', e o Relojoeiro compreende que o esquecimento, neste contexto, não é apenas a perda da memória, da identidade, da história pessoal, mas algo muito mais radical, muito mais absoluto, uma exclusão ontológica, um apagamento não apenas da existência, mas da própria possibilidade de existir, como se o desertor fosse removido não apenas do presente, mas de todo o passado e futuro, como se nunca tivesse nascido, como se nem mesmo a possibilidade de seu nascimento tivesse existido.

A decisão de ajudá-la não é consciente, não é calculada, não é resultado de uma deliberação racional, de uma análise de custos e benefícios, de uma ponderação ética sobre o certo e o errado, é um impulso tão inevitável quanto a queda de uma pedra liberada no ar, tão natural quanto a expansão do universo, tão inquestionável quanto o próprio fluxo do tempo que ele começou a questionar, o Relojoeiro sabe, com uma certeza que transcende a razão, que não pode abandoná-la neste lugar onde o ``ser'' foi dissociado do ``tornar-se'', neste reino de perfeição estéril onde nada morre porque nada verdadeiramente vive, neste simulacro de existência onde a ordem absoluta revelou-se como a mais sutil forma de caos, como a mais completa negação da possibilidade, da criatividade, da novidade que torna a vida não apenas suportável mas desejável, não apenas desejável mas necessária.

Trama uma fuga, um desafio à ordem imutável do vale, um ato de rebeldia cósmica que sabe, desde o início, estar condenado ao fracasso, pois como pode uma criatura do tempo, um ser definido pela finitude, pela temporalidade, pela mudança, desafiar a própria negação desses princípios? como pode o mortal confrontar o eterno em seu próprio território, usando as armas do contingente contra o necessário, do efêmero contra o permanente? e no entanto, essa consciência da futilidade não diminui sua determinação, pelo contrário, a alimenta, a intensifica, porque há uma dignidade essencial na tentativa, no esforço, na recusa em aceitar o inaceitável, mesmo quando ou especialmente quando o fracasso é a única conclusão possível, há uma nobreza no desespero que reconhece sua própria condição sem ser definido por ela, que age apesar da certeza da derrota, que afirma a liberdade precisamente quando ela parece mais ilusória.

Observa os ciclos repetitivos, as rotinas imutáveis, os padrões ossificados que governam a existência no vale, e começa a perceber que a imutabilidade não é realmente uma ausência de tempo, mas um tempo preso em um loop perfeito, um eterno retorno não no sentido nietzschiano de uma repetição cósmica onde cada evento, cada experiência, cada pensamento recorre infinitamente, mas no sentido muito mais restrito, muito mais empobrecido, de um único momento, de uma única configuração do possível que se repete sem variação, sem evolução, sem propósito além da própria perpetuação, é o tempo reduzido à sua expressão mais básica, mais primitiva, mais esquemática, despojado de tudo o que o torna significativo, que o torna fértil, que o torna humano.

É precisamente essa repetição que revela a possibilidade, a única possibilidade, de escape, pois o que se repete com perfeição absoluta torna-se, paradoxalmente, previsível, e o que é previsível pode ser antecipado, e o que pode ser antecipado pode, talvez, ser subvertido, não pela força, não pela velocidade, não pela astúcia, mas pela introdução deliberada de uma variável no sistema perfeito, de uma perturbação no campo de força estático, de um ruído na sinfonia monótona da igualdade, e essa variável, essa perturbação, esse ruído tomará a forma de algo tão simples, tão básico, tão fundamental que os arquitetos da imutabilidade, em sua obsessão pela complexidade, pela totalidade, pela perfeição, talvez o tenham negligenciado completamente: o acaso, a contingência, o imprevisto, não como conceitos abstratos, mas como realidades concretas, como potências do ser.

Na calada da noite, embora ``noite'' seja apenas uma convenção nestas coordenadas onde a luz parece ter uma qualidade constante, imutável, sem as variações circadianas que marcam o ritmo da vida na superfície terrestre, o Relojoeiro e a mulher começam a correr, não como quem foge, mas como quem cria, como quem gera a própria possibilidade de fuga pelo ato mesmo de fugir, cada passo é uma batalha contra a inércia, contra a entropia negativa que caracteriza este lugar onde a ordem não emerge do caos, mas o precede, o determina, o impossibilita, cada respiração é uma afronta à imutabilidade, cada batimento cardíaco uma pequena revolução contra a tirania da permanência, e há algo de glorioso nessa rebelião condenada, nessa insurreição impossível, nesse gesto quixotesco contra moinhos de vento que são, na verdade, deuses disfarçados, leis cósmicas mascaradas de fenômenos meteorológicos.

Conseguem sair do vale, e o Relojoeiro sabe que não deveriam ter sido capazes de fazê-lo, que sua fuga bem-sucedida é uma contradição, uma impossibilidade lógica, uma quebra na estrutura ficcional que sustenta este universo de regras implacáveis, de leis invioláveis, de necessidades absolutas, e no entanto, aqui estão, ofegantes, exaustos, vivos de uma forma que não eram, que não poderiam ser, dentro dos domínios da imutabilidade, mas não sem consequências, pois o Vale, ofendido, parece retaliar, o tempo ao redor deles distorce-se, acelera e desacelera em um frenesi caótico, como uma substância elástica que, tendo sido esticada além de seus limites naturais, agora contrai-se violentamente, indiscriminadamente, arrastando consigo tudo o que encontra em seu caminho, o Relojoeiro, com seu humor ácido, pensa que o tempo é mais caprichoso do que uma criança mimada, mais vingativo do que um amante traído, mais imprevisível do que um bêbado com as chaves de um carro esportivo.

Finalmente, encontram-se fora do vale, em uma terra onde o tempo volta a fluir, mas de forma errática, instável, como se ainda estivesse se recuperando do trauma de ter sido negado, suprimido, aprisionado, a mulher, livre pela primeira vez, sorri com uma alegria que o Relojoeiro jamais havia visto, uma alegria tão pura, tão fundamental, tão absoluta que dói contemplá-la diretamente, como dói olhar para o sol sem proteção, ``obrigada,'' diz ela, ``você me devolveu a vida, o verdadeiro tempo'', e o Relojoeiro compreende, com uma clareza que só vem nos momentos de extrema lucidez, de extrema receptividade, que vivemos não para acumular experiências, não para colecionar realizações, não para construir legados, mas para estes raros, preciosos, irreplicáveis instantes de autenticidade absoluta, estes lampejos de significado que iluminam, por um breve momento, a noite escura do absurdo, da contingência, da finitude que é a condição humana.

\bigskip

E assim, caro leitor, voltamos ao início, ao ponto onde o Relojoeiro, agora com uma nova compreensão e uma nova companheira, reabre as cortinas de sua oficina. Mas algo mudou. A flor murcha, um símbolo da cidade invertida, agora está ao lado de um fragmento de rocha do vale, um lembrete constante de que o tempo é uma dança entre mudança e constância, entre o movimento e a imobilidade, e não seria essa, afinal, a natureza última da realidade? não essa ou aquela polaridade, não esse ou aquele extremo, mas a tensão dinâmica entre opostos, o diálogo perpétuo entre forças contrárias, a síntese contínua que nunca se resolve completamente, que nunca alcança um estado final, que nunca chega a uma conclusão definitiva, porque o momento em que a dialética se completa, em que todas as contradições são resolvidas, em que todas as perguntas encontram suas respostas, é o momento em que o tempo para, em que a história termina, em que a vida, como a conhecemos e a valorizamos, simplesmente deixa de ser?

Era uma manhã como outra qualquer, ou assim parecia ao Relojoeiro, que, ao abrir as cortinas de sua oficina, deixou que os primeiros raios de sol invadissem o espaço repleto de relógios e sombras. Mas algo dentro dele havia mudado, uma inquietação, um formigamento de pensamentos que não se aquietavam, que não se contentavam com o tic-tac monótono dos ponteiros. E foi então que ele decidiu, com a mesma certeza com que se sabe que o dia segue a noite, que era hora de buscar respostas, era hora de entender o tempo, não como um conjunto de engrenagens e mostradores, mas como a essência que permeia tudo, que define a vida e a morte, o amor e a perda, o encontro e a despedida.

E assim, caro leitor, continuamos nossa jornada, sabendo que cada volta ao início é uma nova camada de compreensão, uma nova peça no quebra-cabeça infinito que é o tempo. Prepare-se, pois o próximo capítulo promete mais mistérios, mais revelações e, claro, mais voltas ao começo. Afinal, o tempo, essa entidade caprichosa, adora um bom enigma, e nós, criaturas que somos do tempo, não podemos deixar de jogar seu jogo, mesmo sabendo que as regras mudam constantemente, que o tabuleiro se reconfigura a cada movimento, que a vitória, se é que existe, consiste não em vencer, mas em continuar jogando, em continuar questionando, em continuar buscando aquela verdade que sempre escapa, que sempre se esconde, que sempre nos convida a avançar mais um passo, virar mais uma página, abrir mais uma porta, mesmo que saibamos, no fundo, que do outro lado encontraremos apenas mais perguntas, mais enigmas, mais labirintos, porque talvez seja esse, afinal, o propósito último da existência: não encontrar, mas buscar; não responder, mas perguntar; não chegar, mas caminhar eternamente em direção a um horizonte que se afasta na mesma medida em que nos aproximamos dele.

\begin{center}
\textit{O fragmento de rocha do Vale da Imutabilidade}
\end{center}


\chapter{O Jardim}

Voltando mais uma vez ao início, nosso Relojoeiro reabre as cortinas da oficina com um olhar que agora mistura a curiosidade de um explorador e o cansaço de um velho marinheiro, de um navegante que atravessou oceanos de possibilidades sem encontrar terra firme, ou talvez tenha encontrado terras demais, ilhas de verdades parciais que não formam um continente, um arquipélago de revelações fragmentadas que recusam organizar-se em um todo coerente, e mesmo assim ele persiste, pois que outra escolha teria, sabendo o que agora sabe, vendo o que agora vê, sentindo o que agora sente, este homem de meia-idade com as mãos calejadas pelo manuseio das horas e o coração arranhado pelas garras invisíveis do tempo, que é tigre e cordeiro, algoz e vítima, veneno e remédio, ele sente que cada volta ao começo o aproxima mais das respostas, ou talvez o empurre mais fundo nas perguntas, mas quem pode distinguir entre aproximação e afastamento quando o próprio sentido de direção foi subvertido, quando o mapa e o território trocaram de lugar, quando quem procura é também o procurado, quando quem pergunta é também a resposta que aguarda ser formulada no silêncio entre as palavras?

Fechando a oficina novamente, ele segue em frente, desta vez em direção ao Jardim do Tempo Perdido, um lugar sobre o qual ouviu murmúrios, fragmentos de conversas, meias frases sussurradas como segredos culposos, um lugar que parece existir nas dobras da realidade, nos interstícios entre o que foi e o que poderia ter sido, nas fronteiras porosas entre a memória e o esquecimento, o conhecido e o irreconhecível, e enquanto caminha, o Relojoeiro sente uma estranha duplicação dentro de si, como se fosse não apenas um, mas muitos, não por fragmentação da personalidade como sugeriria algum diagnóstico apressado, mas por expansão, por multiplicação, como se cada passo o desdobrasse em versões alternativas que pensam pensamentos ligeiramente diferentes, sentem sentimentos sutilmente distintos, perguntam-se perguntas quase idênticas mas com entonações variadas, não sou apenas um, pensa ele, sou uma multidão, um parlamento de eus que debate sem chegar a consenso, uma democracia interna onde cada voz tem direito a voto mas nenhuma tem poder de veto, e não seria essa a melhor definição do que chamamos consciência, esse diálogo perpétuo entre as muitas máscaras que usamos não para esconder, mas para revelar diferentes aspectos do que somos ou poderíamos ser?

Ao se aproximar do jardim, o Relojoeiro nota uma mudança no ar, uma densidade que carrega o peso dos momentos passados, uma substância quase palpável que parece resistir a seus movimentos como água resistiria a um nadador, mas não é água, não é ar, não é matéria no sentido convencional, é algo mais sutil e ao mesmo tempo mais denso, mais essencial, como se o próprio espaço fosse saturado de temporalidade condensada, de passados cristalizados, de possíveis jamais realizados que, no entanto, mantêm uma existência espectral, fantasmática, nas margens do que chamamos real, árvores altas e retorcidas se erguem como guardiãs de segredos antigos, seus troncos marcados não por anéis de crescimento, mas por cicatrizes de eventos que nunca ocorreram, por incisões feitas por facas que nunca foram forjadas, por golpes desferidos por mãos que nunca nasceram, e ainda assim tão reais quanto qualquer coisa neste mundo de aparências enganosas, de certezas ilusórias, de verdades provisórias.

Um aroma de nostalgia permeia o ambiente, não um perfume ou um cheiro que poderia ser analisado por químicos, decomposto em moléculas, reduzido a fórmulas, mas uma presença que se dirige diretamente à alma, ao núcleo mais íntimo e indecifrável do ser, como se cada inspiração trouxesse não apenas oxigênio para os pulmões, mas também fragmentos de vidas não vividas para a consciência, estilhaços de sonhos abandonados para o coração, resíduos de esperanças frustradas para o espírito, o Relojoeiro atravessa o portão enferrujado, que range como se protestasse contra qualquer intruso, contra qualquer testemunha, como se o jardim preferisse permanecer não observado, não visitado, fechado em sua própria completude auto-suficiente, em seu próprio ciclo perpétuo de preservação e transformação, de lembrança e esquecimento, de morte e renascimento.

``Bem-vindo ao Jardim do Tempo Perdido,'' diz um velho jardineiro, materializado tão repentinamente que parece ter brotado do solo como uma de suas plantas, um homem cujas mãos calejadas conhecem o toque delicado do tempo, cujos dedos parecem capazes de sentir não apenas a textura das folhas e o pulso da seiva, mas também a consistência mais sutil das horas abandonadas, dos instantes descartados, dos momentos que escorregaram para fora da consciência coletiva como água entre os dedos, ``aqui cultivamos aquilo que foi esquecido, os momentos que passaram sem deixar rastro, mas que, de alguma forma, continuam a existir'', e o Relojoeiro, este homem que dedicou sua vida a medir o mensurável, a contar o contável, a dividir o divisível, sente uma vertigem diante dessa inversão fundamental, dessa agricultura do impalpável, dessa jardinagem do invisível, pois como se pode cultivar o esquecido quando o próprio ato de cultivo já implica em lembrar?

``E o que faz com essas memórias?'', pergunta o Relojoeiro, mais por educação do que por curiosidade, embora uma parte dele, aquela parte que nunca se satisfaz com explicações superficiais, com respostas convencionais, com verdades de segunda mão, esteja genuinamente intrigada, genuinamente estimulada por este paradoxo ambulante, este homem que colhe o que outros descartam, que preserva o que outros rejeitam, que valoriza precisamente aquilo que a sociedade, em sua obsessão pelo novo, pelo atual, pelo imediato, considera desprezível, dispensável, descartável, e o Relojoeiro pensa, não sem certa ironia, que este jardineiro é, à sua maneira, um revolucionário mais radical do que todos os insurgentes políticos, todos os visionários sociais, todos os reformadores culturais, pois desafia não apenas um sistema específico, mas a própria lógica do tempo linear, a própria tirania do presente sobre o passado.

``Cuidamos delas,'' responde o jardineiro, com a simplicidade de quem explica o óbvio, o auto-evidente, o inquestionável, ``para que nunca desapareçam completamente. Cada flor, cada árvore, representa um instante que alguém esqueceu, mas que o tempo, caprichoso como é, decidiu preservar'', e enquanto fala, suas mãos acariciam uma flor azul de formato impossível, uma estrutura de pétalas que parece desafiar as leis da geometria, da biologia, da física, uma criação que não poderia existir no mundo exterior, no mundo governado pela razão, pela causalidade, pela entropia, mas que aqui, neste jardim onde as regras são outras, onde a lógica é outra, onde a verdade é outra, floresce com uma vitalidade quase obscena, quase ultrajante, como um grito de cor em um universo monocromático, como uma afirmação de possibilidade em um reino de necessidades.

Caminhando pelo jardim, o Relojoeiro encontra árvores cujas folhas sussurram histórias esquecidas, histórias de amores não declarados, de palavras não ditas, de gestos não realizados, de caminhos não tomados, de vidas não vividas, cada folha uma narrativa completa em miniatura, um mundo possível compactado em clorofila e celulose, em luz solidificada e tempo cristalizado, ele para diante de uma pequena árvore que, curiosamente, parece familiar, como se a tivesse visto antes em um sonho, em uma memória, em uma premonição, ou talvez seja apenas o reconhecimento daquilo que sempre esteve dentro dele sem que soubesse, daquilo que sempre foi parte de sua estrutura mais profunda, de seu código mais íntimo, de sua verdade mais essencial.

``Esta é a sua árvore,'' diz o jardineiro, como quem revela um segredo que não deveria ser revelado, uma verdade que não deveria ser expressa, um conhecimento que deveria permanecer implícito, tácito, não articulado, ``cada folha é uma memória que você perdeu, cada ramo é uma parte de sua vida que você esqueceu'', e o Relojoeiro sente um arrepio percorrer sua espinha, uma corrente elétrica que conecta o cérebro às extremidades, o pensamento à sensação, o abstrato ao concreto, pois como pode existir uma representação física, material, tangível daquilo que, por definição, é ausente, é lacuna, é vazio? como pode o esquecido ser preservado sem deixar de ser esquecido? como pode a ausência ser tornada presença sem deixar de ser ausência?

O Relojoeiro toca uma folha, um gesto simples, banal, cotidiano, o tipo de contato com a natureza que normalmente não provocaria mais do que um registro sensorial efêmero, uma impressão tátil momentânea, uma experiência trivial rapidamente substituída por outra igualmente trivial na procissão interminável de estímulos que constitui a consciência ordinária, e no entanto, ao tocar esta folha específica, neste jardim específico, neste momento específico, ele é imediatamente transportado para uma tarde de verão de sua infância, um dia de risos e descobertas, um tempo que ele pensava estar perdido para sempre, mas que agora floresce diante de seus olhos com uma nitidez dolorosa, com uma clareza cortante, com uma imediatez que desafia as fronteiras entre passado e presente, entre memória e experiência, entre o que foi e o que é.

Ele vê-se novamente como criança, não como uma lembrança convencional, não como uma reconstrução mental baseada em fragmentos preservados, em narrativas repetidas, em fotografias desbotadas, mas como uma realidade alternativa, como uma bifurcação da linha temporal onde aquele momento nunca deixou de acontecer, onde continua ocorrendo em um eterno presente, em um agora expandido que contém todos os instantes simultaneamente, ele está de pé em um campo de centeio dourado como o sol, ao lado de seu avô, este homem que o introduziu aos mistérios dos mecanismos, às maravilhas das engrenagens, às magias da medição, o homem cujas mãos, tão semelhantes às suas próprias agora, lhe ensinaram que consertar relógios não era apenas um ofício, mas uma forma de comunhão com o tempo, uma maneira de participar do fluxo universal que conecta todas as coisas, todos os seres, todos os momentos em uma teia infinita de relações recíprocas.

``Veja,'' diz o avô, apontando para o horizonte onde o céu e a terra se encontram em uma linha tão definida que parece ter sido traçada com régua e tinta, ``o tempo não é o que pensamos que é. Não é uma flecha, não é uma linha, não é uma estrada. O tempo é um oceano, e nós somos apenas gotas nesse oceano, temporariamente separadas, destinadas a retornar'', e o menino que o Relojoeiro foi escuta com aquela atenção absoluta, aquela receptividade total que só as crianças possuem, antes que a educação as ensine a filtrar, a categorizar, a hierarquizar, a duvidar, a menino acredita porque ainda não aprendeu a desacreditar, aceita porque ainda não foi condicionado a rejeitar, compreende porque ainda não adquiriu a sofisticada ignorância que chamamos de conhecimento.

A experiência é ao mesmo tempo reconfortante e desconcertante, como encontrar um tesouro há muito perdido apenas para descobrir que ele nunca esteve realmente perdido, apenas temporariamente inacessível, temporariamente obscurecido por camadas de experiências posteriores, de conhecimentos adquiridos, de certezas construídas, o Relojoeiro percebe que a memória não é apenas uma função do cérebro, um processo neurológico, um fenômeno bioquímico, mas uma entidade viva, uma planta que precisa ser cuidada para não murchar e desaparecer, um organismo que respira, que cresce, que evolui, que se adapta, que resiste à entropia não por negá-la, mas por incorporá-la em sua própria natureza, em sua própria estrutura, em sua própria essência.

``Mas por que manter essas memórias vivas?'', pergunta ele, ainda atordoado pela experiência, ainda tentando reorientar-se em um universo que acaba de revelar-se muito mais complexo, muito mais misterioso, muito mais rico do que jamais suspeitara, um universo onde o descartado não desaparece, onde o perdido não se perde, onde o esquecido não se esvai, mas encontra um refúgio, um santuário, um jardim onde pode continuar existindo sob outra forma, sob outra manifestação, sob outra modalidade, ``não seria mais fácil deixá-las morrer?'', e esta pergunta, formulada com a ingenuidade de quem ainda pensa em termos de facilidade, de conveniência, de praticidade, revela o quanto ele ainda está preso ao paradigma da eficiência, ao modelo da utilidade, à lógica do custo-benefício que governa o mundo exterior, o mundo da produção e do consumo, o mundo da acumulação e do descarte.

O jardineiro sorri, um sorriso que carrega a sabedoria de séculos, um sorriso que transcende o individual, o pessoal, o biográfico, para expressar algo universal, algo arquetípico, algo que pertence não a um homem específico, mas à própria humanidade em sua jornada através do tempo, em sua luta contra o esquecimento, em sua resistência à dissolução, ``porque as memórias são a alma do tempo. Sem elas, o tempo é apenas uma sequência de eventos vazios, um desfile de horas sem significado. Preservamos as memórias para lembrar que cada momento, por mais insignificante que pareça, é parte de quem somos'', e nestas palavras simples, nesta declaração direta, nesta afirmação desprovida de ornamentos retóricos, de complexidades sintáticas, de sofisticações conceituais, o Relojoeiro encontra uma verdade tão fundamental, tão essencial, tão inescapável que se pergunta como pôde viver tantos anos sem reconhecê-la, sem honrá-la, sem incorporá-la à sua compreensão de si mesmo e do mundo.

Enquanto o Relojoeiro absorve essas palavras, ele vê uma mulher que parece estar perdida em suas próprias lembranças, uma figura de contornos suaves e expressão distante, como alguém que habita simultaneamente múltiplos planos de existência, múltiplas camadas de realidade, múltiplas dimensões de ser, ela olha para ele com uma mistura de tristeza e reconhecimento, como se estivesse vendo não apenas o homem que ele é agora, mas também o menino que ele foi, o adolescente que ele foi, o jovem que ele foi, e talvez também o velho que ele será, todas essas versões coexistindo em um único ponto focal, em um único momento de percepção expandida que transcende a linearidade, a sequencialidade, a unidirecionalidade do tempo convencional.

``Eu conheço você,'' diz ela, e sua voz tem a qualidade etérea de uma música ouvida em sonho, de uma conversa recordada através de paredes, de um eco que chega antes do som que o originou, ``nos encontramos aqui muitas vezes, mas você sempre esquece'', e nesta aparente contradição, neste paradoxo verbal, neste koan espontâneo, o Relojoeiro vislumbra uma possibilidade tão radical, tão revolucionária, tão transformadora que sua mente inicialmente recua, resiste, rejeita, porque aceitá-la significaria reconfigurar completamente sua compreensão não apenas do tempo, não apenas da memória, não apenas da consciência, mas da própria natureza da realidade, da própria estrutura do ser, da própria arquitetura do possível.

``Como posso esquecer algo tão importante?'', pergunta o Relojoeiro, intrigado por esta inversão da lógica habitual, por esta subversão da causalidade normal, por esta reorientação da sequência costumeira onde a importância de um evento é proporcional à sua memorabilidade, onde o significativo é, por definição, o que não pode ser esquecido, o que resiste ao desgaste temporal, o que permanece na consciência apesar da erosão contínua provocada pela passagem das horas, dos dias, dos anos, das décadas, e há em sua pergunta não apenas curiosidade intelectual, não apenas interesse conceitual, não apenas desejo de compreensão, mas também um elemento de angústia existencial, de inquietação ontológica, de desconforto metafísico diante da possibilidade de que aquilo que consideramos mais essencial, mais definidor, mais constitutivo de nossa identidade seja precisamente o que está mais sujeito ao esquecimento, à perda, à dissolução.

``Porque este é o Jardim do Tempo Perdido,'' responde ela, como se isso explicasse tudo, como se nessas poucas palavras estivesse contida toda a cosmologia, toda a epistemologia, toda a metafísica necessária para compreender as leis que governam este lugar, este espaço-tempo anômalo, este enclave de diferentes regras, diferentes princípios, diferentes realidades, ``e aqui, tudo que é importante tende a ser esquecido, para que possamos redescobrir constantemente'', e há nesta formulação uma poesia tão profunda, uma verdade tão condensada, uma sabedoria tão cristalina que o Relojoeiro sente como se algo dentro dele estivesse sendo simultaneamente destruído e reconstruído, desfeito e refeito, aniquilado e regenerado, como se suas estruturas internas, seus mapas conceituais, seus esquemas perceptivos estivessem sendo não apenas revisados ou atualizados, mas fundamentalmente transformados, metamorfoseados em algo completamente novo, completamente diferente, completamente outro.

Com essas palavras ecoando em sua mente, o Relojoeiro decide que é hora de partir, não porque tenha compreendido tudo, não porque tenha assimilado completamente a lição deste lugar estranho, este locus de contradições e paradoxos, este território onde as regras habituais da lógica, da causalidade, da temporalidade são simultaneamente observadas e subvertidas, mas porque intui que a próxima etapa de sua jornada o aguarda, que o próximo fragmento do quebra-cabeça existencial que tenta montar se encontra em outro lugar, que o próximo nível de compreensão exige uma nova paisagem, um novo contexto, um novo conjunto de desafios e revelações, agradece ao jardineiro e à mulher, prometendo a si mesmo que nunca mais esquecerá o que aprendeu ali, mesmo sabendo, com uma parte mais sábia de sua consciência, que tais promessas são tão inevitáveis quanto seu eventual descumprimento, tão necessárias quanto sua inescapável violação.

Com passos lentos, ele deixa o jardim, carregando consigo uma folha da sua árvore, um lembrete constante de que o tempo é feito de memórias, e que preservar essas memórias é preservar a própria essência da vida, esta mínima porção de matéria vegetal, esta estrutura celular aparentemente insignificante, este fragmento de clorofila e celulose torna-se, em suas mãos, em sua percepção, em sua consciência, algo infinitamente mais valioso do que qualquer jóia, qualquer metal precioso, qualquer artefato raro, pois contém não apenas um momento específico, uma experiência particular, uma vivência singular, mas o princípio mesmo da temporalidade significativa, da duração qualitativa, da continuidade essencial que transcende a mera sucessão de instantes, a simples acumulação de agoras, a pura adição de segundos, minutos, horas.

\bigskip

E assim, caro leitor, voltamos ao início, ao ponto onde o Relojoeiro, agora com uma folha de memória nas mãos e um coração mais leve, reabre as cortinas de sua oficina, e este movimento, este gesto aparentemente banal de revelar o interior de um espaço anteriormente oculto, ganha agora uma dimensão simbólica que antes lhe escapava, uma profundidade de significado que antes não percebia, uma ressonância cósmica que antes não sentia, porque agora ele sabe, com uma certeza que nenhuma dúvida pode abalar, que cada vez que abre uma cortina está também abrindo uma memória, cada vez que deixa a luz entrar está também deixando o passado retornar, cada vez que ilumina um espaço está também ressuscitando um tempo que parecia morto mas que apenas dormia, esperando ser acordado, esperando ser lembrado, esperando ser vivido novamente.

\begin{center}
\textit{A folha do Jardim do Tempo Perdido}
\end{center}


%%%%%%%%%%%%%%%%%%%%%%%%%%%%%%%%%%%%%%%%%%%%%%%%%%%%%%%%%%%%%%%%%%%%%
% PARTE III - A REVELAÇÃO
%%%%%%%%%%%%%%%%%%%%%%%%%%%%%%%%%%%%%%%%%%%%%%%%%%%%%%%%%%%%%%%%%%%%%

\part{A Revelação}

\chapter{O Teatro}

Voltando ao início mais uma vez, o Relojoeiro reabre as cortinas da oficina, agora com uma expressão que mistura cansaço e uma determinação quase teimosa, seu rosto iluminado pela luz oblíqua da manhã como um palco sob holofotes amarelados enquanto atores invisíveis se posicionam nas asas, aguardando sua deixa para entrar e pronunciar falas que, embora pareçam espontâneas, foram escritas muito antes de nascerem, muito antes mesmo que a ideia do teatro existisse, muito antes que a primeira palavra fosse pronunciada pela primeira garganta humana, ele sente que cada volta ao começo o empurra mais fundo em uma espiral de descobertas e perplexidades, cada retorno não é mera repetição, mas um novo começo que carrega em si a memória de todos os começos anteriores, e assim, fechando a oficina novamente, ele segue em direção ao próximo destino: o Enigma Filosófico do Tempo, um lugar onde as certezas vão para morrer e as dúvidas se multiplicam como coelhos tirados da cartola de um mágico embriagado pela própria ilusão.

\begin{center}
\textsc{Coro:} \textit{O tempo circular! O eterno retorno! A serpente que devora a própria cauda!}
\end{center}

Ao se aproximar do local, o Relojoeiro percebe que entrou em um terreno onde a lógica foi abandonada em favor de debates intermináveis e reflexões que poderiam enlouquecer qualquer mente sensata, como se tivesse atravessado não apenas um limite geográfico, mas uma fronteira epistemológica, um limiar entre modos de conhecer e estar no mundo, o cenário à sua frente assemelha-se a um anfiteatro grego despojado de ornamentos supérfluos, com degraus de pedra branca dispostos em semicírculo ao redor de um espaço central onde figuras se movem com a deliberada lentidão de quem carrega o peso de séculos de pensamento, cada gesto uma citação, cada postura uma referência a filósofos há muito desaparecidos mas ainda presentes na teia invisível das ideias que sustenta o mundo, e o Relojoeiro, observador relutante deste espetáculo conceitual, hesita entre avançar e recuar, entre participar e testemunhar, entre ser ator e espectador neste drama sem resolução.

E por falar em mentes sensatas, ele se depara com um grupo de filósofos, cada um mais excêntrico que o outro, discutindo com a paixão de adolescentes que acabaram de descobrir a filosofia. Como se o destino não fosse já irônico o suficiente, jogando-o de um extremo a outro do espectro temporal, da cidade onde o tempo corre ao contrário até o vale onde ele sequer se move, e agora\ldots{} Agora está cercado por aqueles que pretendem explicá-lo! Talvez seja mais fácil consertar um relógio sem jamais tê-lo aberto do que entender o tempo ouvindo aqueles que mais falam e menos vivem.

``Bem-vindo ao nosso humilde círculo de tormento intelectual,'' diz um deles, um homem alto e magro, com um olhar que poderia perfurar uma parede e encontrar do outro lado não tijolos ou argamassa, mas as formas puras das ideias platônicas flutuando no vazio primordial que precede toda materialidade, toda temporalidade, toda existência corporificada, ``aqui discutimos o tempo, não como ele é, mas como ele poderia ser, deveria ser, ou nunca será'', e neste momento preciso, como se respondendo a um sinal invisível, os outros filósofos interrompem suas próprias conversas para formar um semicírculo ao redor do recém-chegado, como corvos reunidos ao redor de uma carcaça promissora, de um cadáver conceitual ainda quente o suficiente para alimentar teorias famintas, especulações vorazes, sistemas insaciáveis.

O primeiro filósofo avança enquanto os outros congelam em posições estilizadas:

``Tempo é uma ilusão, uma construção da mente para dar sentido ao caos da existência! O que chamamos de passado não existe mais, o que chamamos de futuro ainda não existe, e o que chamamos de presente é tão infinitesimalmente pequeno que escapa à apreensão antes mesmo de ser reconhecido! Somos prisioneiros de uma ficção que nós mesmos criamos!''

Do lado oposto, outro filósofo responde com veemência apaixonada:

``Não, não! Tempo é uma constante, uma linha reta que todos seguimos desde o nascimento até a morte! É a única verdade objetiva em um universo de incertezas, a única realidade imutável em um cosmos de transformações! Negar o tempo é negar a própria existência, é refugiar-se em um solipsismo covarde que recusa enfrentar a marcha inexorável dos segundos!''

Uma figura andrógina vestida com tecidos fluidos que parecem mudar de cor surge entre eles --- o Místico:

``Ah, o tempo é ambos, e nenhum, e tudo mais! O tempo é uma espiral infinita, onde passado, presente e futuro se entrelaçam em uma dança eterna! É simultaneamente o dançarino e a dança, o observador e o observado, a pergunta e a resposta! Vocês tentam capturá-lo com palavras, mas é como tentar aprisionar o vento em uma gaiola de ossos!''

O Relojoeiro, pressionado contra a parede metafórica do anfiteatro pelo peso das afirmações categóricas, das certezas proclamadas, das verdades declamadas com a convicção inabalável dos que nunca duvidaram realmente, dos que nunca enfrentaram a vertigem do não-saber, do não-compreender, do não-apreender, sente uma irritação crescer dentro dele, não a irritação superficial de quem tem sua paciência testada por inconveniências menores, mas a irritação profunda, existencial, quase sagrada de quem vê o mistério que devotou sua vida a compreender ser reduzido a fórmulas pré-fabricadas, a citações de segunda mão, a posições irreconciliáveis mais interessadas em contrapor-se umas às outras do que em aproximar-se da verdade, se é que tal coisa existe fora dos sistemas auto-referenciais que chamamos de filosofia, de ciência, de religião, de arte.

``E se o tempo for apenas uma desculpa para não enfrentarmos a realidade?'' --- explode o Relojoeiro, interrompendo o fluxo de argumentos. Todos congelam e o encaram. ``Uma desculpa para não vivermos plenamente, para adiarmos decisões e sonhos? Vocês falam e falam e falam, constroem teorias sobre teorias, castelos conceituais tão distantes do chão que o ar rarefeito já lhes afetou o cérebro! Enquanto isso, o tempo --- esse suposto objeto de seus estudos --- escorre por entre seus dedos como água, deixando-os secos, vazios, com nada além de palavras mortas para mostrar por uma vida inteira de especulações inúteis!''

O silêncio que se segue é denso, como se o ar tivesse se transformado em chumbo, como se as moléculas de oxigênio, nitrogênio e argônio tivessem subitamente adquirido a massa de elementos mais pesados, mais sólidos, mais permanentes, os filósofos olham para o Relojoeiro com uma mistura de espanto e admiração, como se ele tivesse dito algo profundamente verdadeiro, ou profundamente tolo, mas eles não conseguem decidir qual das duas coisas é, talvez porque a fronteira entre a verdade e a tolice seja tão tênue, tão porosa, tão arbitrária quanto a fronteira entre o tempo e a eternidade, entre o ser e o nada, entre a palavra e o silêncio.

\begin{center}
\textsc{Coro:} \textit{(sussurrando) A ingenuidade da prática! A arrogância da experiência! A sabedoria do não-saber!}
\end{center}

O Místico quebra o silêncio, aproximando-se com movimentos fluidos e um sorriso enigmático:

``Então talvez tenhamos começado a entender\ldots{}'' --- toca levemente o braço do Relojoeiro, um contato que parece transmitir não apenas calor físico mas também algo mais sutil, mais essencial, como se alguma substância invisível fluísse de um corpo para outro, de uma consciência para outra. ``O tempo não é algo que possamos definir, apenas sentir. E na sensação, encontrar nossa própria verdade.''

Enquanto fala, um objeto cristalino materializa-se em sua mão, como se condensado do próprio ar. O Místico o oferece ao Relojoeiro:

``Leve isto. É um fragmento do agora. Não do seu agora ou do meu agora, mas do Agora que contém todos os presentes possíveis. Ele reflete a luz de maneira diferente a cada momento, para lembrar-lhe que a verdade, como o tempo, nunca é fixa, nunca é definitiva, nunca é completa.''

Com essas palavras, o debate parece ter chegado a um impasse, ou talvez a uma conclusão, mas quem pode dizer com certeza? O Relojoeiro sente que algo mudou, não nos outros, mas em si mesmo, uma recalibração interna, um ajuste sutil nas engrenagens de sua compreensão, como se a agulha de uma bússola que por muito tempo apontou para um norte magnético tivesse subitamente redescoberto o norte verdadeiro, reorientando não apenas o instrumento, mas o próprio navegante, o próprio oceano, o próprio conceito de direção, ele levanta-se, agradece com um aceno, e decide que é hora de partir, não porque tenha obtido respostas definitivas --- percebe agora que tais respostas não existem, ou se existem, não podem ser apreendidas por mentes confinadas no tempo linear, em consciências limitadas pela sequencialidade, pela causalidade, pela narratividade --- mas porque compreendeu que sua busca não deveria ser por certezas imutáveis, por verdades eternas, por conhecimentos absolutos, mas pela sabedoria de reconhecer que o tempo é menos sobre medição e mais sobre vivência, menos sobre cronômetros e mais sobre emoções, menos sobre passado/presente/futuro e mais sobre a qualidade única, irrepetível, insubstituível de cada experiência.

O Relojoeiro reflete consigo mesmo: ``Saí de lá com mais perguntas do que respostas, mas com uma nova percepção: talvez o tempo seja menos sobre medição e mais sobre vivência, menos sobre cronômetros e mais sobre emoções. Posso passar a vida inteira tentando definir o tempo e morrer sem compreendê-lo, ou posso aceitar sua natureza paradoxal e viver plenamente cada momento --- mesmo sabendo que essa plenitude é tão ilusória quanto duradoura.''

Os filósofos, um por um, se retiram para as sombras. Apenas o Místico permanece.

``Lembre-se: não é o tempo que passa, somos nós que passamos pelo tempo. Ele estava aqui antes de nascermos e continuará depois que partirmos. Nossa única verdadeira possessão é a qualidade de nossa atenção a cada instante.''

``E se eu esquecer esta lição?''

``Então você a reencontrará, sob outra forma, em outro lugar, em outro tempo. Pois nada que é verdadeiramente importante pode se perder --- apenas se transformar.''

\bigskip

E assim, caro leitor, voltamos ao início, ao ponto onde o Relojoeiro, agora com a mente cheia de novos enigmas, reabre as cortinas de sua oficina, e desta vez não é apenas a luz do sol que invade o espaço repleto de relógios e sombras, mas uma nova compreensão, uma nova perspectiva, uma nova relação com o mistério que tentou decifrar e que, paradoxalmente, só começou a entender quando aceitou que talvez não possa ser completamente decifrado, que talvez sua indecifrabilidade seja precisamente sua essência, sua verdade, sua beleza.

A oficina do Relojoeiro se ilumina. Ele coloca o cristal ao lado dos outros objetos colecionados: a flor murcha da cidade invertida, o fragmento de rocha do vale e a folha do jardim.

``A inversão do tempo, que nos ensina que a direção não é destino\ldots{} A imobilidade do tempo, que nos mostra que a permanência não é eternidade\ldots{} A memória do tempo, que nos revela que o esquecido nunca está verdadeiramente perdido\ldots{} E agora, o enigma do tempo, que nos lembra que compreender não é o mesmo que definir. Cada resposta que encontro é apenas o disfarce de uma nova pergunta. Cada certeza conquista o direito de tornar-se uma dúvida mais profunda.''

\begin{center}
\textsc{Coro:} \textit{O tempo é um rio que corre ao contrário\ldots{} O tempo é uma montanha imóvel no horizonte\ldots{} O tempo é um jardim onde as memórias florescem\ldots{} O tempo é um enigma sem solução, uma pergunta sem resposta, um espelho que reflete não o que somos, mas o que estamos sempre nos tornando!}
\end{center}

E assim, caro leitor, continuamos nossa jornada, sabendo que cada volta ao início é uma nova camada de compreensão, uma nova peça no quebra-cabeça infinito que é o tempo. Prepare-se, pois o próximo capítulo promete mais mistérios, mais revelações e, claro, mais voltas ao começo. Afinal, o tempo, essa entidade caprichosa, adora um bom enigma, e nós, marionetes desse teatro cósmico, continuamos a dançar conforme a música que nem mesmo percebemos estar ouvindo, atores de um drama cujo autor talvez seja apenas nossa própria necessidade de dar sentido ao insensato, forma ao informe, nome ao inominável.

\begin{center}
\textsc{Fim do Quinto Ato}

\textit{O cristal do Místico}
\end{center}


\chapter{A Praça}

Voltando ao início mais uma vez, o Relojoeiro reabre as cortinas da oficina, agora com uma determinação quase fanática, travessia permanente, terceiras margens cruzando, como se estivesse prestes a enfrentar o próprio Tempo em um duelo de morte --- \textit{nonada!} ---, os objetos têm sombras que não lhes pertencem e o ar circula como uma guilhotina invisível entre passado e futuro, ele sente que cada volta ao começo o empurra mais fundo em espiral de descobertas e perplexidades, atravessando, ele atravessa, todas as coisas atravessa, cada retorno diferente do anterior como rio que nunca é o mesmo, nem para o rio nem para quem nele mergulha duas vezes, o Tempo é Sertão, vasto, traiçoeiro, sem começo definido nem fim verdadeiro, território onde Deus e o Diabo se encontram para negociar as horas. Fechando a oficina novamente, trancafio de mundos, ele segue em direção ao próximo destino: a Revolução dos Ponteiros, um lugar onde o tempo é questionado, desafiado e, finalmente, subvertido, como criança que desmonta brinquedo para entender seu funcionamento e o destrói exatamente por isso.

A cidade se descortina aos poucos, fragmentada, objetos cortados da realidade com tesouras cegas, cacos de espelho refletindo pedaços de céu que nunca existiram, árvores feitas de fumaça, nuvens feitas de pedra, as ruas se contorcem como víboras feridas, edifícios inclinados para todos os lados exceto para cima, e o Relojoeiro olho-que-tudo-vê e nada compreende, o Relojoeiro sente um estranhamento visceral, um desenraizamento da própria certeza de existir, pois até mesmo a ordem espacial habitual foi deformada, distorcida, virada do avesso e costurada novamente com linhas feitas de intervalos, a cidade é menos um lugar e mais uma pergunta, os tijolos das casas são interrogações empilhadas, as janelas são reticências abertas para um céu que talvez seja apenas o vazio tingido de azul, a Revolução, ele descobre, não teve início em um momento específico --- \textit{nonada que era!} --- sempre esteve acontecendo, espiral que se fecha e se abre simultaneamente, cobra engolindo a própria cauda enquanto continua crescendo.

Ao se aproximar do local, o Relojoeiro nota uma agitação que contrasta violentamente com a calma estática do Vale da Imutabilidade, lugar-não-lugar ficado para trás, mundo-sumiço nas veredas das possibilidades, aqui o tempo não é apenas desafiado, mas atacado com fúria revolucionária, marretadas no vidro protetor dos relógios de parede, fogueiras alimentadas com calendários, ampulhetas quebradas derramando areia que, em vez de cair, sobe como poeira rebelde desafiando a gravidade, os habitantes, homens e mulheres em cujos rostos o Relojoeiro vê simultaneamente juventude e decrepitude, inocência e sabedoria, como se suas identidades fossem colagens malfeitas, retalhos de épocas diversas costurados à força em um único semblante, têm olhares intensos e gestos decididos, preparando-se para uma revolução que promete libertar o tempo de suas correntes e, quem sabe, libertar a eles mesmos, da sina de esperar, de transformar a vida em espera, cada segundo consumindo o próximo como num rosário de contas invisíveis.

``Tempo não corre mais, parado está, esperando que o libertemos dele mesmo'' --- voz da mulher cortava a tarde como faca corta fruta madura --- ``Vamos tomar a torre do relógio! Vamos quebrar as correntes do tempo e mostrar ao mundo que somos donos do nosso próprio destino!'' --- crepitação de folha seca esmagada, futuro pisoteado antes de nascer. A multidão, movendo-se como criatura única com mil cabeças, avança em direção à torre central que se ergue na praça, um obelisco monstruoso cujo relógio gigante projeta sombra de prisão circular sobre as ruas, sobre os rostos, sobre as almas de todos os que vivem sob sua autoridade, sob seu tique-taque perpétuo que é grilhão invisível, respirador artificial de uma sociedade moribunda que respira apenas porque o relógio lhe concede essa permissão, um segundo de cada vez, um minuto de cada vez, uma hora de cada vez, nada mais, nada além, apenas a medida exata de vida necessária para continuar funcionando, para continuar servindo, para continuar existindo sem jamais realmente viver.

``Bem-vinda ao coração da revolta,'' diz uma jovem de olhos flamejantes e voz apaixonada, colares feitos de engrenagens quebradas, pulseiras de cordas de relógio arrebentadas, o cabelo preso com pequenos ponteiros, como se houvesse incorporado à própria aparência os despojos de seus inimigos vencidos, ``aqui não aceitamos o tempo como um tirano. Estamos prontos para quebrar os ponteiros e redefinir nossas vidas.'' A jovem estende a mão, um convite, uma iniciação, um pacto de sangue com o futuro incerto que pretendem criar, e em sua palma o Relojoeiro vê linhas que não formam os padrões habituais, não há linha da vida, linha do coração, linha da cabeça, apenas espirais concêntricas que parecem pulsar como se tivessem vida própria, como se fossem pequenos vórtices tentando sugar quem os observa para uma dimensão desconhecida, para um território além do mapeável, além do concebível, além do tempo.

O homem que foi o Relojoeiro que talvez já não o seja mais, pois as identidades se dissolvem nesta cidade como açúcar em água quente, com sua habitual ironia que é escudo e lança, armadura e ferida, pensa em responder que talvez eles estivessem confundindo libertação com anarquia, que talvez uma revolução contra o tempo seja tão impossível quanto uma revolta contra a gravidade, contra a morte, contra a própria condição de ser finito em um universo onde apenas a finitude parece ter alguma realidade tangível, mas contém-se, engole as palavras que teriam saído como arame farpado de sua garganta, em vez disso, ele observa, olho-tempo que tudo observa para depois relatar a estranhos que não existem, pois às vezes observar é a melhor maneira de entender o caos antes de mergulhar nele, a cabeça cheia de sussurros de outros lugares visitados antes, fragmentos de certezas que se dissolvem, tesouras do regime cortando as últimas fotos de família, o passado perseguido pelos cães da censura.

``Vamos tomar a torre do relógio!'' grita um líder carismático, cuja voz reverbera pela praça central como trovão num céu sem nuvens, uma contradição materializada, uma impossibilidade tornada real pela pura força de vontade, pela pura intensidade de propósito, pela pura convicção de que o impossível é apenas o possível que ainda não encontrou ocasião de manifestar-se, ``Vamos quebrar as correntes do tempo e mostrar ao mundo que somos donos do nosso próprio destino!'', e sua voz é semente em terra fértil, é faísca em palha seca, é pedra na superfície de lago plácido, gerando ondas concêntricas que se propagam, ampliam, multiplicam, até que não há um único coração na praça que não pulse no mesmo ritmo, não há um único peito que não respire a mesma indignação, não há uma única alma que não abrigue o mesmo sonho de liberdade.

O Relojoeiro se vê arrastado pela multidão como folha em redemoinho, verá-tinha que segue o fluxo universal, sentindo-se como um turista em uma guerra civil, um observador passivo em uma revolução que não compreende totalmente, as pessoas ao seu redor se movem com uma determinação alimentada pela raiva, pelo desespero, pela esperança, pela possibilidade de um amanhã diferente, um amanhã que não seja apenas mais um dia idêntico aos anteriores, marcado pela mesma sucessão monótona de horas, pela mesma progressão impiedosa de segundos que devoram a vida sem oferecer nada em troca, nem significado, nem propósito, nem consolação, vemo-nos naquilo que nos olha, olho de vida olhando olho da morte, espelho partido refletindo o rosto da transgressão --- \textit{``eu queria e não queria, ah, mas queria sim''} ---, sabe que deveria tentar escapar, retornar à segurança relativa de sua condição de observador, mas há algo nessa luta que ressoa dentro dele, algo que desperta uma chama adormecida, uma brasa esquecida sob as cinzas frias da rotina, da aceitação, da resignação que por tanto tempo confundiu com sabedoria.

A multidão avança como uma maré irresistível, humana-criatura com membros feitos de indignação e torsos feitos de esperança, e o Relojoeiro, agora parte dela, parte absorvida pelo todo sem perder completamente a individualidade, como gota de chuva que se junta ao oceano sem deixar de ser água, se encontra diante da torre do relógio, um monstro de pedra que parece zombar de seus esforços, com suas paredes maciças, suas janelas estreitas como olhos semicerrados de um gigante observando com desdém os pequenos seres que ousam desafiá-lo, seu relógio no topo, tão alto que seria necessário torcer o pescoço até o limite para enxergá-lo completamente, cada número romano uma sentença, cada movimento dos ponteiros um decreto, cada tique-taque uma chicotada nas costas do tempo subjugado que, apesar de ser a própria substância da existência, foi aprisionado, domesticado, escravizado por sua própria criação, por sua própria manifestação, por sua própria prole.

``Derrubem-no!'' grita o líder, e com uma determinação feroz, os revolucionários atacam o símbolo do tempo tirânico, alguns com martelos, outros com picaretas, alguns com as próprias mãos sangrando, dilacerando-se contra a pedra impassível que parece absorver a violência, a fúria, o desespero sem mostrar o menor sinal de fraqueza, de dano, de cedência. O Relojoeiro, o homem que já se reconhece nos outros como se todos fossem um, intrépido buscador de respostas, empunhando uma barra de ferro que encontrou no chão, herança retorcida de alguma construção anterior, de algum projeto abandonado, de algum sonho interrompido, golpeia o relógio com toda a força de suas convicções recém-descobertas, com todo o peso de suas dúvidas antigas, com toda a energia de sua busca interminável por uma verdade que talvez seja impossível de alcançar, que talvez não exista de forma alguma, que talvez seja apenas o horizonte sempre distante que nos mantém em movimento, que nos impede de desistir, de aceitar, de conformar-nos com as respostas fáceis, com as verdades convenientes, com as certezas confortáveis que são o ópio dos que temem o desconhecido.

Cada golpe é uma afirmação, um grito de liberdade, um desafio ao inevitável tic-tac que dita suas vidas --- batendo, pancada, pedaço por pedaço, lá-é-aqui que atravessa os vivos e os mortos ---, o pisar pesado dos revolucionários fazendo estremecer o chão, a vibração subindo pelas paredes como uma premonição sísmica, como um aviso de que algo fundamental está prestes a mudar, a romper-se, a transformar-se em algo diferente, algo novo, algo ainda sem nome porque o vocabulário do futuro ainda não foi inventado, ainda não foi sonhado, ainda não foi sequer imaginado como possibilidade concreta. \textit{``Assim falei, assim é''} --- a realidade dobra-se quando se ousa falar o que ainda não existe, nomeação é criação, verbo é fertilização da vastidão do nada.

Os golpes, primeiro descoordenados, caóticos, individuais, começam a sincronizar-se, a unificar-se em um único ritmo, um único impacto, uma única vontade manifestada através de centenas de braços, de centenas de ferramentas, de centenas de intenções que, fundidas, tornam-se uma força capaz de alterar a própria estrutura da realidade, de perfurar a membrana do possível para alcançar o impossível que espera do outro lado, ansioso por nascer, por materializar-se, por existir além da mera conceitualização, além da mera potencialidade, além da mera especulação teórica. \textit{``É neste tempo-sem-tempo que existimos verdadeiramente''}, diz uma voz, ou talvez um pensamento, ou talvez um eco do futuro viajando de volta para testemunhar seu próprio nascimento.

Uma primeira fissura aparece na base da torre, tênue como um fio de cabelo, insignificante como o primeiro momento de dúvida em uma fé até então inabalável, desprezível como a primeira gota de água que escapa da represa aparentemente impenetrável, mas ela está lá, visível para quem sabe olhar, para quem aprendeu que as grandes transformações começam sempre nos detalhes mais ínfimos, nas alterações mais sutis, nas mudanças quase imperceptíveis que, num piscar de olhos, num suspiro do universo, num segundo distraído do destino, tornam-se avalanches, inundações, revoluções que redefinem o curso da história, o fluxo do tempo, a natureza mesma da existência. \textit{``O grande é pequeno e o pequeno é grande, depende de onde se vê, vendo por onde''} --- fala o Relojoeiro para quem não ouve, palavras apenas para si mesmo, única audiência verdadeira de todos os nossos pensamentos.

A fissura cresce, ramifica-se como veias num corpo vivo, estende-se para cima, para os lados, explorando as fragilidades ocultas da estrutura, os pontos onde a pedra é menos densa, menos resistente, menos capaz de sustentar o peso não apenas da própria torre, mas de tudo o que ela representa, de todo o sistema que ela sustenta, de toda a ordem que ela impõe sobre um mundo que, em sua natureza mais básica, mais fundamental, mais essencial, talvez não deva ser ordenado, talvez não possa ser contido, talvez resista intrinsecamente a qualquer tentativa de domesticação, de categorização, de medição. O relógio no topo começa a titubear, os ponteiros tremem como agulhas de bússola próximas a um campo magnético potente, oscilando entre o norte verdadeiro e o norte distorcido, entre o tempo convencional e o tempo revolucionário, entre o chronos que devora seus filhos e o kairos que oferece redenção.

Enquanto o relógio desmorona, engrenagem por engrenagem, parafuso por parafuso, peça por peça, vidro estilhaçado refletindo centenas de futuros possíveis em cada fragmento, algo dentro do Relojoeiro também se quebra, fissuras dividindo o que antes fora sólido, o Olho transforma em palavras, as palavras em objetos, os objetos voltam ao vazio --- milagre ao contrário, tempo redimido de sua própria tirania ---, se quebra também no revolucionário que foi por alguns instantes, quando incorporou a fúria coletiva, a indignação compartilhada, a esperança comum de uma nova ordem, uma nova lógica, uma nova relação com o tempo e, por extensão, com a própria vida. Ele percebe que a revolução não é apenas contra o tempo, mas contra as limitações que impõem a si mesmos, contra as fronteiras artificiais que separam o possível do impossível, contra a divisão arbitrária entre o que é e o que poderia ser, se tivéssemos a coragem, a visão, a determinação de transcender as categorias herdadas, as definições recebidas, as estruturas impostas que confinam tanto o pensamento quanto a experiência em corredores estreitos, em caminhos predeterminados, em rotas aprovadas que levam apenas aos destinos sancionados, nunca às paisagens inexploradas onde novas formas de existência nos aguardam.

Em meio aos destroços, ele encontra o líder carismático, agora olhando para o vazio com um misto de triunfo e desespero, de realização e perplexidade, de vitória e desorientação, os olhos fixos no espaço antes ocupado pela torre, na ausência que agora é mais significativa, mais poderosa, mais transformadora do que qualquer presença poderia ser, o silêncio do relógio destruído mais eloquente do que seu anterior tique-taque perpétuo, o vazio deixado pela estrutura mais substancial do que a própria estrutura jamais foi em sua materialidade limitada, em sua existência circunscrita, em sua realidade finita.

``Conseguimos,'' diz o líder, falante que fala para fazer real o sonho recém-nascido, mágico que pronuncia encantamento para concretizar a ilusão, criança que nomeia o invisível para torná-lo palpável, mas sua voz não tem a certeza de antes, o timbre vacila entre a afirmação e a interrogação, a entonação oscila entre a declaração e o questionamento, o volume flutua entre o anúncio e o sussurro, como se a vitória tão longamente sonhada, tão ardentemente desejada, tão ferrenhamente perseguida, agora que finalmente foi alcançada, revelasse sua natureza ambígua, sua essência contraditória, sua qualidade paradoxal de ser simultaneamente culminação e começo, conclusão e introdução, resposta e pergunta. ``E agora, o que faremos?'', e esta é a verdadeira questão revolucionária, a única que realmente importa, a única que contém em si o germe de todas as outras perguntas, de todas as outras dúvidas, de todas as outras possibilidades, pois qualquer revolução que saiba exatamente o que fazer após a vitória não é revolução verdadeira, é apenas substituição, apenas troca, apenas permutação de uma ordem por outra, de uma estrutura por outra, de uma limitação por outra.

O Relojoeiro, com a sabedoria irônica de quem já viu demais, de quem já experimentou demais, de quem já questionou demais para aceitar qualquer resposta como definitiva, qualquer verdade como absoluta, qualquer caminho como único, responde: ``Agora, precisamos aprender a viver sem as correntes, sem os ponteiros. A liberdade não está em destruir o tempo, mas em vivê-lo plenamente.'', palavras-mapa desenhando território novo, nomeação de espaço que surge apenas quando é nomeado, ``Precisamos aprender a sentir a vida como um rio que corre entre sertões e não como uma flecha disparada para o alvo predestinado. Livrar-se do tirano é apenas o primeiro passo; o verdadeiro desafio é não se tornar outro tirano no processo. Por isso a revolução nunca termina --- atravessa e é atravessada pelo próprio movimento que a origina.'' A boca fala porque as palavras estavam lá antes dele, sussurradas por algum antepassado que viu o tempo desdobrar-se em múltiplas dimensões.

O líder olha para ele, verdadeiramente olha, não apenas vê mas enxerga, não apenas registra mas compreende, não apenas observa mas reconhece, e em seu olhar há a luz do entendimento, da percepção súbita, da compreensão repentina que, como um raio em noite escura, ilumina brevemente a paisagem inteira, revelando conexões, relações, padrões invisíveis sob a luz normal, sob a percepção rotineira, sob a consciência habitual, ``Você fala como alguém que já viu muitas revoluções,'' diz ele, não como acusação, não como suspeita, não como confrontação, mas como reconhecimento de um igual, de um companheiro, de um aliado na interminável luta contra as simplificações, contra as reduções, contra as banalizações que são a verdadeira tirania, a verdadeira opressão, a verdadeira prisão da qual todos devemos tentar escapar.

Enquanto a poeira assenta e o caos começa a se organizar em uma nova ordem, não imposta de cima, não decretada por autoridade, não estabelecida por decreto, mas emergente do próprio processo de interação, de colaboração, de coexistência entre indivíduos livres, entre consciências autônomas, entre seres autodeterminados, o Relojoeiro percebe que sua jornada ainda não terminou, que o quebra-cabeça ainda não está completo, que a tapeçaria ainda não está tecida em sua totalidade, que o mapa ainda tem áreas em branco, regiões inexploradas, territórios desconhecidos que aguardam seu olhar, suas mãos, seu coração, sua mente para serem integrados à cartografia sempre expansiva, sempre inclusiva, sempre inacabada da compreensão. Ele agradece ao líder e aos revolucionários, deixando-os para reconstruir suas vidas em um novo ritmo, um ritmo que eles mesmos escolheram, um ritmo que não lhes é imposto por autoridade externa, por estrutura alienante, por sistema opressivo, mas que emerge de seus próprios corpos, de suas próprias necessidades, de seus próprios desejos, de suas próprias aspirações por uma existência mais autêntica, mais plena, mais verdadeira.

\bigskip

E assim, caro leitor, voltamos ao início, ao ponto onde o Relojoeiro, agora com as mãos sujas de poeira e o coração inflamado de novas convicções, reabre as cortinas de sua oficina --- \textit{``Mas o reabrir é um fingimento do abrir pela primeira vez? Ou o primeiro abrir já era repetição de outro mais antigo?''} --- pensamento que atravessa seus neurônios como relâmpago. Mas algo havia mudado. Ao lado da flor murcha da cidade invertida, do fragmento de rocha do vale, da folha do jardim e do cristal do Místico, agora repousa um pedaço do relógio destruído, um símbolo da revolução que não só desafiou o tempo, mas o reinventou, uma engrenagem torcida como uma pequena galáxia metálica, um fragmento de vidro que reflete não apenas a luz, mas todas as possibilidades que existem entre o ser e o não-ser, entre o acontecer e o não-acontecer, entre o lembrar e o esquecer, entre viver e deixar de existir.

Era uma manhã como outra qualquer, ou assim parecia ao Relojoeiro, que, ao abrir as cortinas de sua oficina, deixou que os primeiros raios de sol invadissem o espaço repleto de relógios e sombras. Mas algo dentro dele havia mudado, uma inquietação, um formigamento de pensamentos que não se aquietavam, que não se contentavam com o tic-tac monótono dos ponteiros. E foi então que ele decidiu, com a mesma certeza com que se sabe que o dia segue a noite, que era hora de buscar respostas, era hora de entender\ldots{}

E assim, caro leitor, continuamos nossa jornada, atravessando de margem a margem, sabendo que cada volta ao início é uma nova camada de compreensão, uma nova peça no quebra-cabeça infinito que é o tempo, cada retorno atravessa e é atravessado, espirais que refletem facetas do mesmo diamante primordial. Tempos são rios diferentes que atravessam o mesmo objeto atravessante. \textit{``O real não está no início nem no fim, ele se dispõe para a gente é no meio da travessia.''} Prepare-se, pois o próximo capítulo promete mais mistérios, mais revelações e, claro, mais voltas ao começo. Afinal, o tempo, essa entidade caprichosa, nonada pulsante, adora um bom enigma, tesouras do destino recortando figuras no papel amarelado dos dias, e nossos nomes se apagando lentamente dos registros, como neve que derrete ao sol da memória coletiva.

\begin{center}
\textit{A engrenagem quebrada da Revolução dos Ponteiros}
\end{center}


%%%%%%%%%%%%%%%%%%%%%%%%%%%%%%%%%%%%%%%%%%%%%%%%%%%%%%%%%%%%%%%%%%%%%
% PARTE IV - O RETORNO
%%%%%%%%%%%%%%%%%%%%%%%%%%%%%%%%%%%%%%%%%%%%%%%%%%%%%%%%%%%%%%%%%%%%%

\part{O Retorno}

\chapter{O Horologista}

Voltando ao início uma última vez, o Relojoeiro reabre as cortinas da oficina, mas as mãos que realizam esse gesto já não lhe parecem próprias, como se pertencessem a outro homem, a outro tempo, a outra existência. A luz que invade o espaço não ilumina mais do que sombras sobre sombras, realidades palimpsestas sobrepostas em camadas de significado que já não consegue separar. Com um olhar que condensa tudo o que viu, tudo o que compreendeu e, principalmente, tudo o que jamais poderá compreender, ele contempla os cinco objetos dispostos sobre sua mesa de trabalho: a flor murcha da Cidade das Horas Invertidas, o fragmento de rocha do Vale da Imutabilidade, a folha do Jardim do Tempo Perdido, o cristal do Místico e a engrenagem quebrada da Revolução dos Ponteiros.

``Não são suficientes'', murmura para si mesmo, palavras que não arranham sequer a superfície do vazio que se expande dentro dele. ``Cinco fragmentos não bastam para reconstruir o universo. Cinco peças não são suficientes para consertar o tempo.''

Toma então uma decisão que já estava escrita no primeiro momento em que abriu as cortinas da oficina, no início desta jornada que só agora compreende não ter sido uma escolha, mas um destino inescapável, uma necessidade inscrita na própria estrutura do ser. Ele recolhe os cinco objetos, guarda-os em um pequeno saco de couro que prende ao cinto, e, pela última vez, fecha a porta da oficina. Desta vez, porém, não guarda a chave no bolso. Deixa-a na fechadura, girando lentamente até ouvir o clique final que separa definitivamente o que foi do que será.

Seu caminho agora é direto, sem hesitações, sem desvios. Não é mais um peregrino tateando no escuro, mas um homem que caminha com a certeza brutal dos que aceitaram seu fado. Seus pés o levam de volta ao Jardim do Tempo Perdido, atravessando paisagens que parecem simultaneamente familiares e estranhas, como se o mundo inteiro tivesse se transformado em um eco, uma reverberação de algo original que já não existe, que talvez nunca tenha existido.

O portão enferrujado range, mais alto agora, um protesto metálico contra sua intrusão, ou talvez um lamento pelo que está prestes a acontecer. O Jardim parece esperá-lo, como se cada folha, cada flor, cada partícula de terra soubesse que este momento era inevitável. O ar está pesado com o aroma das memórias, mais denso do que da primeira vez, quase sufocante em sua intensidade.

O jardineiro está lá, cuidando de uma pequena muda que parece ter sido plantada recentemente, suas raízes ainda não completamente estabelecidas, sua existência ainda precária, suspensa entre o ser e o não-ser. Ao ver o Relojoeiro, ele não demonstra surpresa, apenas ergue os olhos cansados e acena como quem reconhece um velho amigo, ou talvez um antigo adversário com quem finalmente fez as pazes.

``Você voltou,'' diz o jardineiro, e não é uma pergunta, mas uma constatação. ``Eu sabia que voltaria.''

``Eu também,'' responde o Relojoeiro, ``embora só agora compreenda que sempre soube.''

A mulher que ele conheceu em sua primeira visita surge entre as árvores, seu rosto uma máscara de eternidade que esconde camadas infinitas de histórias não contadas, de vidas não vividas. Ela se aproxima sem pressa, seus passos tão leves que não perturbam sequer o mais delicado fio de orvalho.

``Nós o esperávamos,'' diz ela, sua voz como o sussurro do tempo passando por folhas secas. ``A árvore o esperava.''

O Relojoeiro segue ambos até sua árvore, aquela mesma que lhe foi mostrada antes, aquela que contém todas as memórias que perdeu, todas as versões dele mesmo que foram esquecidas, abandonadas ao longo do caminho. Mas a árvore está diferente agora. Parece maior, mais robusta, como se na sua ausência tivesse continuado a crescer, alimentada por novas memórias, por novos esquecimentos.

``Durante sua jornada,'' explica o jardineiro, ``você perdeu mais de si mesmo, e tudo isso foi parar aqui. Cada revelação que teve, cada verdade que vislumbrou, cada certeza que abandonou, tudo isso gerou novas folhas, novos ramos, novas raízes que se aprofundaram no solo do tempo.''

O Relojoeiro se aproxima da árvore, toca seu tronco com a reverência de quem toca um altar. Sob seus dedos, a madeira parece pulsar com uma vida própria, um batimento que ressoa com seu próprio coração, como se a separação entre ele e a árvore fosse apenas uma ilusão, uma convenção arbitrária que já não tem mais sentido, que já não tem mais propósito.

``Eu compreendo agora,'' diz o Relojoeiro, e sua voz tem o timbre de quem fala não apenas para os presentes, mas para todos os que vieram antes e todos os que virão depois, para o próprio tecido do tempo que se estende infinitamente em todas as direções. ``Não vim buscar respostas. Vim oferecer-me como resposta.''

Ele retira do pequeno saco de couro os cinco objetos, símbolos das cinco verdades parciais que descobriu em sua jornada, e os dispõe cuidadosamente ao redor da base da árvore, formando um círculo perfeito, uma mandala de significados fragmentados que, juntos, compõem uma imagem ainda incompleta, ainda insuficiente, mas que aponta para algo além de si mesma, para uma totalidade que não pode ser contida em formas finitas, em conceitos limitados, em palavras inadequadas.

``Veja,'' diz ele, apontando para a flor murcha, ``em um mundo onde o tempo flui ao contrário, a beleza está no declínio, não no florescimento. A perfeição não é um estado a ser alcançado, mas uma condição a ser transcendida.''

Toca então o fragmento de rocha, seus dedos mapeando as estrias, as fissuras, as cicatrizes deixadas por uma eternidade comprimida em matéria. ``Em um mundo onde nada muda, a própria imutabilidade se torna uma prisão. A permanência absoluta é tão insuportável quanto a transformação incessante.''

Recolhe a folha de sua árvore, aquela que trouxe consigo em sua primeira visita, agora mais seca, mais frágil, quase translúcida em sua delicadeza. ``Em um mundo onde as memórias são preservadas, descobrimos que o esquecimento é tão necessário quanto a lembrança. É no espaço entre recordar e esquecer que habitamos, que existimos, que somos.''

Ergue o cristal contra a luz, permitindo que seus múltiplos facetas refratem cores que não têm nome, que não podem ser categorizadas, que existem apenas no limiar da percepção. ``Em um mundo de especulações filosóficas, aprendemos que toda verdade é parcial, toda certeza provisória, todo conhecimento limitado. A sabedoria não está em saber, mas em reconhecer os limites do saber.''

Finalmente, toma entre os dedos a engrenagem quebrada, seus dentes torcidos, sua forma distorcida pela violência da revolução. ``E em um mundo de rebelião, compreendemos que destruir não é suficiente. Que qualquer revolução contra o tempo é também uma revolução dentro do tempo, limitada pelas mesmas estruturas que pretende transcender.''

Olha então para o jardineiro e para a mulher, seus rostos agora claramente o mesmo rosto visto de ângulos diferentes, de tempos diferentes, como se fossem não indivíduos distintos, mas manifestações da mesma consciência observando-se a si mesma através do prisma fraturado da temporalidade.

``Eu nasci no tempo,'' diz o Relojoeiro, sua voz agora adquirindo a qualidade sonora de uma declaração final, de um testemunho definitivo, como se cada palavra fosse esculpida não em ar, mas em uma substância mais duradoura, mais essencial. ``Cresci medindo-o, dividindo-o, vendendo-o em parcelas cuidadosamente reguladas. Acreditei ser seu mestre quando era apenas seu servo. Pensei controlá-lo quando era por ele controlado.''

Ele caminha lentamente ao redor da árvore, seus olhos registrando cada detalhe, cada nuance, cada variação de textura e cor, como se estivesse memorizando um rosto amado antes de uma separação que sabe ser final, irrevogável, absoluta.

``Busquei compreender o tempo, dominá-lo, transcendê-lo, e em cada etapa desta jornada, perdi um pouco mais de mim mesmo. Ou talvez, em cada perda, tenha me aproximado um pouco mais do que realmente sou, daquilo que sempre fui antes de me fragmentar em segundos, minutos, horas.''

Sua voz agora adquire uma qualidade metálica, uma ressonância que faz vibrar não apenas o ar, mas o próprio solo sob seus pés, as próprias raízes da árvore que se estendem profundamente na terra, conectando-se com todas as outras árvores, com todas as outras memórias, com todas as outras existências.

``Como Antígona diante de Creonte, recuso-me a obedecer às leis impostas pelos tiranos do tempo linear. Como ela, escolho honrar uma lei mais antiga, mais profunda, mais verdadeira. Não a lei promulgada por deuses ou homens, mas a lei inscrita na própria estrutura do ser, na própria natureza da existência.''

O jardineiro e a mulher permanecem imóveis, testemunhas silenciosas de uma transformação que parece não apenas inevitável, mas necessária, não apenas necessária, mas desejada, não apenas desejada, mas eternamente presente como possibilidade aguardando sua realização, sua manifestação, sua concretização.

``Vós que viveis na ilusão do tempo sequencial, que vos agarrais à segurança ilusória do antes e depois, do causa e efeito, da origem e destino, escutai-me!'', sua voz agora é um trovão, uma tempestade, um cataclismo que sacode os alicerces do jardim, que faz tremer as próprias raízes da realidade. ``Não existe passado nem futuro, apenas o eterno presente que contém em si todas as possibilidades, todas as realizações, todas as existências!''

Os cinco objetos dispostos em círculo começam a vibrar, a ressoar em frequências impossíveis, emitindo sons que não pertencem ao espectro audível, cores que não pertencem ao espectro visível, energias que não podem ser medidas, quantificadas, categorizadas por instrumentos fabricados dentro dos limites do tempo linear.

``Eu, que fui relojoeiro, que medi e dividi o indivisível, que fragmentei o contínuo, que aprisionei o infinito em caixas de segundos, minutos e horas, renuncio a este ofício! Renuncio a esta ilusão! Renuncio a esta tirania que impus a mim mesmo e aos outros!''

E com estas palavras, ele abraça o tronco da árvore, sentindo contra seu peito o pulsar lento, constante, eterno da seiva que flui por suas veias vegetais, da vida que se manifesta não em unidades discretas, em momentos separados, em instantes isolados, mas em um fluxo contínuo, ininterrupto, indivisível de pura existência.

``Aqui permaneço,'' declara, ``não como prisioneiro do tempo, mas como seu libertador. Não como vítima da eternidade, mas como sua testemunha. Não como servo do chronos, mas como manifestação do kairos, do momento oportuno, da plenitude do instante que contém em si toda a eternidade.''

E enquanto fala, algo extraordinário começa a acontecer. A fronteira entre sua carne e a madeira da árvore começa a dissolver-se, não como uma transformação física, mas como uma revelação da verdade que sempre esteve ali, oculta apenas pela ilusão da separação, pela falsa dicotomia entre observador e observado, entre sujeito e objeto, entre ser e tempo.

``Para vós, que ainda habitais o reino do tempo fracionado, parecerá que me tornei parte desta árvore, que abandonei minha humanidade para fundir-me com ela. Mas a verdade é mais simples e mais complexa: sempre fomos um só. Eu e esta árvore, eu e minhas memórias, eu e meus esquecimentos, eu e o tempo, todos aspectos da mesma realidade indivisível que apenas aparenta dividir-se por conveniência, por limitação, por incapacidade de apreender o todo em sua vertiginosa totalidade.''

O jardineiro se aproxima, seus olhos agora claramente os mesmos olhos do Relojoeiro, apenas mais antigos, mais sábios, mais resignados à inevitabilidade do ciclo eterno que não é repetição, mas renovação; que não é retorno, mas reinvenção; que não é círculo fechado, mas espiral ascendente.

``Bem-vindo ao lar,'' diz ele, e sua voz é simultaneamente a voz do jardineiro e a voz do Relojoeiro, a voz do presente e a voz do futuro, a voz do que é e a voz do que será. ``Você finalmente completou o ciclo.''

``Não,'' responde o Relojoeiro, sua voz agora indistinguível do sussurro das folhas movidas por uma brisa que parece surgir do interior da própria árvore, do interior do próprio tempo. ``Não completei o ciclo. Transcendi-o. Não há ciclo a ser completado porque não há separação a ser superada. O início e o fim são ilusões, pontos arbitrários em uma circunferência infinita.''

A mulher também se aproxima, e em seus olhos o Relojoeiro vê todos os que amou, todos os que perdeu, todos os que esqueceu e todos os que o esqueceram, todos unidos não por laços temporais, não por relações causais, mas pela própria natureza da existência que é, em sua essência mais profunda, relação, conexão, interdependência.

``Eles virão,'' diz ela, referindo-se a outros que, como o Relojoeiro, eventualmente encontrarão o caminho até o jardim, até a árvore, até a compreensão que transcende a compreensão. ``Outros Relojoeiros, outros Buscadores, outros Peregrinos do Tempo. E você estará aqui para recebê-los, como parte desta árvore, como parte deste jardim, como parte desta verdade que não pode ser dita, apenas vivida.''

``Sim,'' concorda o Relojoeiro, sua voz agora mais fôlego que som, mais pensamento que palavra, mais silêncio que discurso. ``Estarei aqui, não como eu mesmo, mas como parte de algo maior, algo mais verdadeiro, algo mais real do que qualquer identidade limitada pelo tempo, qualquer consciência confinada à sequencialidade, qualquer existência aprisionada na linearidade.''

E com estas palavras, a fusão se completa, não como um fim, mas como um reconhecimento, uma aceitação, uma celebração da verdade que sempre esteve presente: o Relojoeiro e a árvore, o observador e o observado, o ser e o tempo, todos aspectos da mesma realidade indivisível que apenas se fragmenta no espelho quebrado da percepção limitada, da consciência condicionada, da linguagem inadequada.

Onde antes havia um homem e uma árvore, agora há apenas a árvore, mais robusta, mais viva, mais radiante em sua silenciosa sabedoria, em sua paciente eternidade, em sua generosa inclusividade que acolhe todas as memórias, todos os esquecimentos, todas as versões possíveis e impossíveis do que foi, do que é, do que será e do que poderia ter sido.

\bigskip

E assim, caro leitor, nossa jornada aparentemente chega ao fim, embora saibamos agora que não há fim, apenas transformação; não há conclusão, apenas continuidade; não há resolução, apenas dissolução das falsas dicotomias, das ilusórias separações, das arbitrárias demarcações que impomos à realidade por incapacidade de abraçá-la em sua desconcertante totalidade.

O tempo, essa entidade caprichosa que tentamos aprisionar em relógios, em calendários, em narrativas com começo, meio e fim, revela-se finalmente não como tirano a ser derrubado, nem como enigma a ser decifrado, mas como o próprio meio em que existimos, o próprio tecido de que somos feitos, a própria substância que somos e que nos constitui.

E talvez, caro leitor, você também seja parte desta árvore, parte deste jardim, parte desta história que não é apenas uma história entre muitas, mas a única história que sempre foi e sempre será contada, a história do tempo que se descobre a si mesmo através de nós, que se experimenta a si mesmo através de nossas alegrias e dores, que se conhece a si mesmo através de nossas buscas e descobertas, que se ama a si mesmo através de nossos amores e perdas.

Talvez, ao fechar este livro, você não esteja realmente terminando uma leitura, mas continuando uma conversa iniciada muito antes de seu nascimento e que prosseguirá muito além de sua morte, uma conversa entre todas as consciências que já existiram, que existem e que existirão, unidas não pelo fio frágil da causalidade linear, mas pela teia robusta da simultaneidade eterna, da coexistência atemporal, da interpenetração infinita de todos os seres, todos os momentos, todas as possibilidades.

Ou talvez --- e esta é a possibilidade mais vertiginosa, mais libertadora, mais transformadora --- não haja ``talvez'', não haja ``você'', não haja ``eu'', não haja ``tempo'', apenas a pulsação eterna do ser que se manifesta como multiplicidade aparente, como diversidade ilusória, como separação temporária, apenas para retornar sempre, inevitavelmente, alegremente, à unidade fundamental, à integridade essencial, à totalidade primordial que é nossa verdadeira natureza, nossa autêntica condição, nossa legítima herança.

E assim, sem fim porque nunca realmente começou, sem conclusão porque nunca realmente se desenvolveu, sem despedida porque nunca realmente nos encontramos, nossa história --- que é a única história, que é todas as histórias --- continua, perpetuamente presente, eternamente atual, infinitamente real no único momento que existe: agora.

\vfill

\begin{center}
\copyright{} 2025 Gustavo Mendes e Silva. Todos os direitos reservados.
\end{center}


\backmatter%%%%%%%%%%%%%%%%%%%%%%%%%%%%%%%%%%%%%%%%%%%%%%%%%%%%%%%

%% Posfácio
\chapter*{Posfácio}

O leitor que chegou até aqui talvez se pergunte: o que aconteceu com o Relojoeiro? A resposta é simultaneamente simples e complexa: ele se tornou o que sempre foi. Não houve transformação, apenas revelação. Não houve fim, apenas reconhecimento de que nunca houve começo.

A jornada do Relojoeiro é, em última análise, a jornada de todos nós. Nascemos no tempo, vivemos no tempo, morreremos no tempo. E no entanto, carregamos dentro de nós uma centelha de eternidade, uma intuição de totalidade que nos impede de aceitar completamente nossa condição finita, nossa existência fragmentada, nossa consciência sequencial.

Esta obra não pretende oferecer respostas. Pretende, isso sim, formular perguntas de maneira mais precisa, mais honesta, mais corajosa. Porque no fim --- se é que existe um fim --- não são as respostas que nos definem, mas as perguntas que ousamos fazer.

\vspace{1cm}
\begin{flushright}
\textit{Gustavo Mendes e Silva}\\
\textit{2025}
\end{flushright}

%% Sobre o Autor
\chapter*{Sobre o Autor}

\textbf{Gustavo Mendes e Silva} é escritor e pensador brasileiro. Suas obras exploram as fronteiras entre ficção e filosofia, narrativa e meditação, literatura e vida. \textit{O Relojoeiro} representa sua tentativa mais ambiciosa de capturar em palavras aquilo que, por natureza, escapa a qualquer captura.

\vfill

\begin{center}
\copyright{} 2025 Gustavo Mendes e Silva. Todos os direitos reservados.
\end{center}

%%%%%%%%%%%%%%%%%%%%%%%%%%%%%%%%%%%%%%%%%%%%%%%%%%%%%%%%%%%%%%%%%%%%%%

\end{document}
